In software development projects, often time and budget costs are exceeded. Laird and Brennan~\cite{Laird:SoftwareMeasurement} investigated in 2006 that 23\% of all software projects are canceled before completion. Furthermore, of the completed projects, only 28\% are delivered on time with the average project overrunning the budget with 45\%. Testing is an important part of software development, because it decreases future maintainance costs~\cite{McConnell:testing}. Testing is a complex process and should be done often~\cite{Pol:testing}. Therefore, the testing process should be as efficient as possible in order to save resources.

Test automation allows repeated testing during the development process. The advantage of this is that bugs are found early and can therefore be fixed early.  A widely used practice is maintaining a \textit{test suite}, which is a collection of test-cases. However, when the creation of a test suite is done manually, this still leaves room for human error~\cite{Blackburn:testing}. The process of deriving tests tends to be unstructured, barely motivated in the details, not reproducible, not documented, and bound to the ingenuity of single engineers~\cite{Utting:MBTTaxonomy}.

\section{Model-based Testing}
The existence of an artifact that explicitly encodes the intended behaviour can help mitigate the implications of these problems. Creating an abstract representation or a \textit{model} of the system is an example of such an artifact. What is meant by a model in this report, is the description of the behavior of a system. Moreover, the term model will be often used to describe transition-based notations, such as finite state machines, labelled transition systems and I/O automata. Statecharts such as UML models are not considered in this report. 

A model can be used to systematically generate tests for the system. This is referred to as \textit{model-based testing}. This leads to a larger test suite in a shorter amount of time than if done manually. These models are created from the specification documents provided by the end-user. These specification documents are 'notoriously error-prone'~\cite{McCabe:testing}. This implies that the model itself needs validation. Validating the model usually means that the requirements themselves are scrutinised for consistency and completeness~\cite{Utting:MBTTaxonomy}.

Tools for automatic test generation already exist. The testing tool developed by Axini\footnote{http://www.axini.nl/} is used for the automatic test generation on \textit{symbolic} models, which combine a state and data type oriented approach. This tool is used in this report and is referred to as Axini Test Manager (ATM). In Utting et al.~\cite{Utting:MBTTaxonomy}, a taxonomy is done on different model-based testing tools:
\begin{itemize}
  \item TorX~\cite{Tretmans:TorX}: accepts behaviour models such as I/O labelled transition systems. A version of this tool written in Java under continuous development is JTorX~\cite{Belinfante:JTorX}.
  \item Spec Explorer\cite{Veanes:SpecExplorer}: provides a model editing, composition, exploration,
and visualization environment within Visual Studio, and can generate offline .NET test suites or
execute tests as they are generated (online).
  \item JUMBL\cite{Prowell:JUMBL}: an academic model-based statistical testing that supports the development of statistical usage-based models using Markov chains, the analysis of models, and the generation of test cases.
  \item AETG\cite{Cohen:AETG}: implements combinatorial testing, where the number of possible combinations of input variables are reduced to a few 'representative' ones.
\end{itemize}

The stakeholders evaluate the constructed model to verify its correctness. However, the visual or textual representation of large models may become troublesome to understand, which is referred to as the model having a low model transparency or high model complexity. Models that are often used are transition-based, i.e. a collection of nodes representing the states of the system connected by transitions representing an action taken by the system. The problem with such models is that a larger number of states decreases the model transparency. We think that low model transparency make errors harder to detect and that it obstructs the feedback process of the stakeholders. Using models with high transparency is therefore essential.

\section{Graph Transformation}
A formalism that claims to have more model transparency is Graph Transformation. The system states are represented by graphs and the transitions between the states are accomplished by applying graph change rules to those graphs. These rules can be expressed as graphs themselves. A graph transformation model of a software system is therefore a collection of graphs, each a visual representation of one aspect of the system. This formalism may therefore provide a more intuitive approach to system modelling than traditional state machines. Graph Transformation and its potential benefits have been studied since the early '70s. The usage of this computational paradigm is best described by the following quote from Andries et al.~\cite{Andries1999}: \begin{quote}Graphs are well-known, well-understood, and frequently used means to represent system states, complex objects, diagrams, and networks, like flowcharts, entity-relationship diagrams, Petri nets, and many more. Rules have proved to be extremely useful for describing computations by local transformations: Arithmetic, syntactic, and deduction rules are well-known examples.\end{quote} An informative paper on graph transformations is written by Heckel et al.~\cite{Heckel2006187}. A quote from this paper: \begin{quote}Graphs and diagrams provide a simple and powerful approach to a variety of problems that are typical to computer science in general, and software engineering in particular.\end{quote}

The graph transformation tool GROOVE\footnote{http://sourceforge.net/projects/groove/} is used to model and explore graph grammars.

\section{Research goals}\label{sec:goals}
The motivation above is given for using graph grammars as a modelling technique. The goal of this research is to create a system for automatic test generation on graph grammars. If the assumptions that graph grammars provide a more intuitive modelling and testing process hold, this new testing approach will lead to a more efficient testing process and fewer incorrect models. The to be designed system, once implemented and validated, provides a valuable contribution to the testing paradigm. The tools GROOVE and ATM are used to create this system.

The research goals are split into a design and validation component:
\begin{enumerate}
    \item \textbf{Design}: Design and implement a system using ATM and GROOVE which performs model-based testing on graph grammars.
    \item \textbf{Validation}: Validate the design and implementation using case studies and performance measurements.
\end{enumerate}

The result of the design goal is one system called the GROOVE-Axini Testing System (GRATiS). The validation goal uses case-studies with existing specifications from systems tested by Axini. Each case-study has a graph grammar and a symbolic model which describe the same system. GRATiS and ATM are used for the automatic test generation on these models respectively. Both the models and the test processes are compared as part of the validation.

The solution has to uphold three requirements:
\begin{enumerate}
\item A graph grammar must be used as the model; it must derive from the specification and be used for the testing.
\item It must be possible to measure the test progress/completion, by means of \texit{coverage} statistics (explained in detail in section~\ref{sec:coverage}).
\item The solution must be efficient: it should be usable in practice, therefore the technique should be scalable and the imposed constraints reasonable from a practical view point.

\section{Roadmap}
This report features five more chapters: first, the concepts described in this chapter are elaborated in chapter \ref{chapter:background}. The technique used in GRATiS is described in chapter \ref{chapter:gg_to_sts}. The setup for the validation of GRATiS is in chapter \ref{chapter:validation}. Chapter~\ref{chapter:model-examples} features model examples on which GRATiS is applied and validated. Chapter~\ref{chapter:case-study} features a case study where GRATiS is applied on an existing software system. Finally, conclusions are drawn in chapter \ref{chapter:conclusion}.
