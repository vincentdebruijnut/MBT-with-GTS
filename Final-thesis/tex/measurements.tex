\section{Measurements}\label{sec:measurements}

This sections lists possible measurements on the models and execution of GRATiS on those models.

\subsection{Simulation}
The STS created by GRATiS and the created STS by hand can be compared. It can be observed whether the STS created by GRATiS simulates the STS created by hand and vice versa. When either is not the case, the models show a different possible behaviour for the SUT. The reasons behind this difference is then explored. HOW? AXEL vragen

\subsection{Performance}
The performance in terms of execution time and 'heap-size' can be measured and compared. Assuming both the STS created by GRATiS and by hand simulate each other, these metrics will be the same for the testing part. Therefore, the execution time and heap size of the STS creation is measured. EXAMPLE FIGURE?

%\subsection{Error detection}
%This measurement introduces an error in the SUT and records how fast the error is found by both models. Again, this measurement is redundant assuming both models simulate each other. This measurement is therefore not included in the analysis.

\subsection{Model redundancy}
It is possible that the generated STS $s$ can be reduced to a smaller STS $t$. It is measured whether $t$ exists such that $\mathit{traces}(s) \subseteq \mathit{traces}(t) \land |\Switches_t| < |\Switches_s|$. HOW? AXEL VRAGEN

\subsection{Model complexity}\label{sec:complexity_measurement}
The complexity of the generated STS and the GG can be measured using Halstead's software science (QUOTE NEEDED: Halstead Elements of Software science 1977). This method is used in measuring software complexity, but can also be used in analyzing model complexity. In Halstead's software science, the operators and operands are counted. The operators are the function symbols, the operands are the identifiers. Both models have nodes and edges. In an STS, the locations are counted as nodes, the switch relations as edges. Nodes and edges are considered to be operands. The distinct number of operators ($n_1$) and operands ($n_2$) are counted as well as the total number of operator occurrences ($N1$) and operand occurrences ($N_2$). These metrics combined lead to the \textit{Volume} of the models. The volume is calculated by: $(N_1+N_2)*\mathit{log}_2(n_1+n_2)$. Comparing the volumes of the STS and GG gives an indication of the relative model complexity.

\subsection{Extendability}
The models can be extended to include more functionality. In this measurement, a realistic scenario is introduced where additional functionality is required. It is then measured how much the complexity increases, using the measurment in section~\ref{sec:complexity_measurement}.
