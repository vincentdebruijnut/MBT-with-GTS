\section{Algorithm}
This section describes an algorithm for creating an STS $J$ from a GG $K$ and an initial graph $\startGraph$. 

\subsection{Point algebra}
We define a \textit{point algebra} $\PointAlgebra$\newnot{symbol:PointAlgebra} to be an algebra with $\forall \sort \in \Sorts . |\mathbb{U}_\PointAlgebra^\sort| = 1$. We define $O$ to be the GTS obtained from exploring $K$ on $\startGraph$, using $\PointAlgebra$.

\subsection{Graphs}
The graphs $\Graphs$ in a GTS are representations of the states of a transition system. In an STS, this representation is done by the locations and the location variables. In order to obtain the locations and location variables of $J$, the graphs of $O$ are used.

\subsubsection{Locations}
For each $\hostGraph \in \Graphs$ of $0$ a location $\location \in \Locations$ of $J$ is created. The location created from the initial graph $\startGraph$ is the initial location $\initialLocation$. We define the function $\mu_\Locations: \Graphs \rightarrow \Locations$ to obtain the location from the graph .

\subsubsection{Location variables}
The location variables $\LocationVariables$ are identified by the $\mathbb{U}$ nodes in $\startGraph$. Concretely, for each node $n \in \mathbb{U}$, a location variable $\variable \in \LocationVariables$ is created and $\variable \mapsto n$ is added to the initialization $\initializationFunction$.

\subsection{Rule transitions}
Rule transitions are a representation of the switch from one system state to another. In STSs, this representation is done by the switch relations.

\subsubsection{Gates}
For each rule $\ggrule \in \Rules$ of $O$ a gate $\gate \in \Gates$ of $J$ is created, with $\mathit{arity}(\gate) = 0$. We define the function $\mu_\Gates: \Rules \rightarrow \Gates$ to obtain the gate from the rule.

\subsubsection{Interaction variables}
Not possible from GG, only GROOVE-GG specific.

\subsubsection{Guards}
Only possible if we define terms in GGs.

\subsubsection{Update mapping}
Only possible if we define terms in GGs.

\subsubsection{Switch relations}
For each rule transition $\ruleTransition \in \RuleTransitions$ a switch relation $\switch \in \Switches$ is created. We define the function $\mu_\Switches: (\hostGraph \xrightarrow{\ggrule} \hostGraph') \rightarrow (\mu_\Locations(\hostGraph)\xrightarrow{\mu_\Gates(\ggrule), \guard, \updateMapping}\mu_\Locations(\hostGraph'))$

