\section{Requirements considerations}
As described in seciton~\ref{sec:goals}, the GRATiS tool needs to fulfill three requirements:
\begin{enumerate}
\item A graph grammar must be used as the model; it must derive from the specification and be used for the testing.
\item It must be possible to measure the test progress/completion, by means of coverage statistics.
\item The solution must be efficient: it should be usable in practice, therefore the technique should be scalable and the imposed constraints reasonable from a practical view point.
\end{enumerate}

\paragraph*{IOGGs} In order to fulfill the first requirement, stimuli and responses have to be obtained from a GG. ATM uses an IOSTS, where the instantiated switch relations represent a stimulus to or a response from the SUT. GGs have no notion of inputs and outputs, therefore IOGGs have to be used as the model formalism. IOGGs can be explored to IOGTSs and the I/O labels of the IOGTS can be used to represent abstract stimuli/responses.

\paragraph*{On-the-fly vs. offline exploration} The exploration of a GG can be done in two ways: \textit{on the fly}, rule transitions are explored only when chosen by ATM, or \textit{offline}, the GG is first completely explored and then sent to ATM. On-the-fly model exploration works well on large and even infinite models. However, coverage statistics cannot be calculated with this technique. The number of states (graphs) and rule transitions the model has when completely explored are not known, so a percentage cannot be derived. As coverage statistics are an important metric, the offline model exploration is chosen for GRATiS.

\paragraph*{Data values} An IOGTS can potentially be infinitely large, due to the range of data values. This is a potential risk for the validation of the system. A model that is more efficient with data values is an STS. The setup of GRATiS is therefore to transform the IOGG directly to an IOSTS. This transformation should be done efficiently to fulfill the third requirement. Note that the second requirement is still met, because location and switch relation coverage can be calculated on the IOSTS.

\paragraph*{Steps} Taking these requirements into account, the method to achieve the goal of model-based testing on GGs is the following three steps:
\begin{enumerate}
\item Create an IOGG by assigning I/O types to the graph transformation rules of a GG
\item Create an IOSTS from the IOGG
\item Perform the model-based testing on the IOSTS
\end{enumerate}
The rest of this chapter describes a declaritive definition for creating an IOSTS from an IOGG.

\section{From IOGG to IOSTS}\label{sec:algorithm}
\subsection{Variables in GGs} 
The location variables in an STS represent an aspect of the modelled system. For instance, if a system keeps track of the number of items in containers, the STS modelling this system could have integer location variables $\mathit{items}_1..\mathit{items}_n$. GGs do not have this kind of variables. The variable nodes in rule graphs are used to match a value in a host graph, which is only available to that rule. A definition of persistent variables in GGs is needed in order to define location variables from an IOGG. We define $\GGVariables$ to be the set of GG variables in a GG.

\vspace{10px}
\begin{definition}\label{def:gg_vars} GG variables \\
A GG variable is a tuple $\langle \variableLabel, \valueLabel\rangle \in \GGVariables$, where:
\begin{itemize}
\item \newnot{symbol:variableLabel}$\variableLabel$ is the \textit{variable label}, which is part of an edge $(\variableAnchor, \variableLabel, \variableAnchor)$, where $\variableAnchor$\newnot{symbol:variableAnchor} is called the \textit{variable anchor}
\item \newnot{symbol:valueLabel}$\valueLabel$ is the \textit{value label}, which is part of a \textit{value edge}: $(\variableAnchor, \valueLabel, \node)$, with $\node \in \mathbb{U}$.
\end{itemize}
\end{definition}
\vspace{10px}

The item/container example modelled in a graph grammar could be a graph node representing a container with an edge labelled `items' to an integer node. This is shown in Figure~\ref{fig:gg_vars_formal}. The variable $(\mathit{var1, items})$ is now represented by this graph. Figure~\ref{fig:gg_vars_groove} shows the same example in GROOVE. The variable edge is omitted here and the variable label is represented by the flag \textit{var1}. This convention for GROOVE is used in the rest of this report.

\begin{figure}[ht]
  \begin{center}
    \subfloat[Formal]{\label{fig:gg_vars_formal}\parbox[b]{4cm}{\centering$\xymatrix{
   \fbox{container} \ar[rr]^{items} \ar@(ul,ur)^{var1} && \fbox{2}
}$
}}
    \subfloat[GROOVE]{\label{fig:gg_vars_groove}\parbox[b]{4cm}{\centering% To use this figure in your LaTeX document
% import the package groove/resources/groove2tikz.sty
%
% Special colors
\begin{tikzpicture}[
% Special color styles
scale=\tikzscale]
\node[node] (n0)  at (0.830, -0.620) {\ml{\textit{var1}\\Container\\items = 2}};
\userdefinedmacro
\end{tikzpicture}
\renewcommand{\userdefinedmacro}{\relax}
}}
  \end{center}
  \caption{Example of a GG variable}
  \label{fig:vars-in-ggs}
\end{figure}

\subsection{Graph exploration with point algebra} 
The IOGG cannot be directly made into an IOSTS, without using the IOGTS. To avoid the problem of a potentially infinitely large IOGTS, the point algebra is used. Using the point algebra when exploring a GG has two effects:
\begin{enumerate}
\item The host graphs that differ only in value nodes are collapsed into one
\item The variable nodes in rule graphs can have at most one image in the host graph
\end{enumerate}
The actual values of the value nodes in host graphs and of the variable nodes in rule graphs are not needed to make the IOSTS, which is explained in the next section. Using the point algebra for the exploration poses some constraints, which are explained in section~\ref{sec:constraints}.

\subsection{The IOGG to IOSTS definition}\label{sec:iogg_to_iosts_definition}
First the declaritive definition is given, then each part of the definition is described in more detail.
\vspace{2px}\begin{definition} IOGG to IOSTS \\
Let $K$ be an IOGG. From $K$ we define an IOSTS $J$. The first step is to explore $K$ using the point algebra $\PointAlgebra$ to an IOGTS $O_\PointAlgebra$. The elements of $J$ can be obtained as follows:
\begin{itemize}
\item $\Locations = \Graphs$
\item $\initialLocation = \startGraph$
\item $\LocationVariables = \GGVariables$
\item $\initializationFunction = \function_{\initializationFunction}$, where $\function_{\initializationFunction}$ is defined below
\item $\Gates = \Rules$
\item $\InteractionVariables = \Variables$
\item $\Switches = \RuleTransitions$, where $\guard, \updateMapping$ are given by functions defined below
\end{itemize}
\end{definition}

\paragraph*{Locations} The locations abstract from data values, exactly like host graphs do under the point algebra. Therefore, the set of locations in $J$ are the set of host graphs in $O_\PointAlgebra$. The initial location in $J$ is also equal to the start graph of $O_\PointAlgebra$.

\paragraph*{Location variables} GG variables were defined to have location variables in GGs. Therefore, it is no surprise that the set of location variables in $J$ is exactly the set of GG variables in $O_\PointAlgebra$. This poses some constraints on creators/erasers for the variable anchors, variable edges and value edges. This is explained in detail in section~\ref{sec:constraints}.

\paragraph*{Initialization function}
The initial value of the location variables is defined by the start graph. The start graph contains the GG variables and their initial values. This poses a constraint on the start graph, which is explained in section~\ref{sec:constraints}. 
\vspace{10px}\begin{definition} Initialization function
\vspace{2px} \\
$\function_{\initializationFunction}:\:(\variableLabel, \valueLabel)\mapsto \node\:|\:(\variableAnchor, \variableLabel, \variableAnchor) \in \DefinedEdges_{\startGraph} \land (\variableAnchor, \valueLabel, \node) \in \DefinedEdges_{\startGraph}$
\end{definition}

\paragraph*{Gates}
The gate of a switch relation represents the stimulus to or response from the SUT. In an IOGG, the rules represent the stimuli and responses. Therefore, the set of gates $\Gates$ is equal to the set of rules $\Rules$.

\paragraph*{Interaction variables}
Interaction variables are used by the gates to represent a stimulus or response variable. The variable nodes in rule graphs are this representation. The set of interaction variables $\InteractionVariables$ is equal to the set of variable nodes $\Variables$. The arity of a rule is defined by the variable nodes in the $\mathit{LHS}$ of the rule:
\vspace{5px} \\
$arity(\ggrule) = |\Variables\cap\DefinedNodes_{LHS}|$

\paragraph*{Switch relations}
A rule transition $\hostGraph \xrightarrow{\ggrule, \rulematch} \hostGraph' \in \RuleTransitions$ is mapped to a switch relation $(\hostGraph\xrightarrow{\ggrule, \function_{\guard}, \function_{\updateMapping}}\hostGraph') \in \Switches$. Note that the set of locations is the set of host graphs, therefore $\hostGraph$ and $\hostGraph'$ represent the source and target location of the switch relation. Also, the set of rules is chosen to be the set of gates, therefore $\ggrule$ represents the gate of the switch relation. The function $\function_{\guard}$ obtains the guard as a term and $\function_{\updateMapping}$ replaces the $\updateMapping$ function. These functions are defined in the next paragraphs.

\paragraph*{Guard}\label{sec:guards}
The guard of a switch relation restricts the use of the switch relation based on the values of the variables. In a GG, a rule is restricted by the terms. The variables used in the terms are interaction variables. Therefore, the first part of the guard is constructed by joining the terms for each term node.
\vspace{10px}\begin{definition} First guard function
\vspace{2px} \\
$\function 1_{\guard}:\: \bigwedge_{\node \in \DefinedNodes_{LHS} \cap \TermNodes,\: t_1 \in \node,\: t_2 \in \node}t_1 = t_2$
\end{definition}
\vspace{10px} 
We apply this to the rule graph in Figure~\ref{fig:guard_example}. This rule graph contains one term node, which in turn contains two terms. Formally, the first part of the guard for this rule graph is:
\vspace{5px} \\
$(x_1+1 = x_1+1) \land (x_1+1 = x_2) \land (x_2 = x_2) \land (x_2 > 3 = x_2 > 3) \land (x_2 > 3 = true) \land (true = true)$
\vspace{5px} \\
When $t_1 = t_2$, the equation for this part is not useful, therefore it is omitted from now on. 

\begin{figure}[h]
  \begin{center}
    $\xymatrix{
   \fbox{$int:x_1$} & \fbox{$r_1$} \ar[l]^{b} \ar[d]^{d} \ar[r]^{a} & \fbox{$r_2$} & \fbox{$x_1+1, x_2$} \\
   & \fbox{$int:x_2$} & \fbox{$x_2 > 3, true$}
}$

  \end{center}
  \caption{a rule graph}
  \label{fig:guard_example}
\end{figure}

This first part of the guard contains only the interaction variables. In an STS, the values for $x_1$ and $x_2$ can be chosen such that the guard holds. However, the variable nodes cannot have just any value; their value is determined by their image in the host graph. This image can be the value of a GG variable. Therefore we add a second part to the guard to link the interaction and location variables.
\vspace{10px}\begin{definition} Second guard function
\vspace{2px} \\
$\function 2_{\guard}:\: \bigwedge_{(\variableLabel, \valueLabel) \in \GGVariables,\: \node \in \Variables \cap \DefinedNodes_{LHS}\: |\: (\variableAnchor, \variableLabel, \variableAnchor) \in \DefinedHostEdges \land (\variableAnchor, \valueLabel, \rulematch(\node)) \in \DefinedHostEdges} (\variableLabel, \valueLabel) = \node$
\end{definition}
\vspace{10px}
We apply this to the rule graph in Figure~\ref{fig:guard_example} and the host graph in Figure~\ref{fig:guard_example_host}. The rule has a match in the host graph. The second part of the guard for this rule is:
\vspace{5px} \\
$(var1,b) = x_1 \land (var1,d) = x_2$

\begin{figure}[h]
  \begin{center}
    $\xymatrix{
   \fbox{$1$} & \fbox{$h_1$} \ar@(ul,ur)^{var1} \ar[l]^{b} \ar[d]^{d} \ar[r]^{a} & \fbox{$h_2$} \\
   & \fbox{$1$}
}$

  \end{center}
  \caption{a host graph}
  \label{fig:guard_example_host}
\end{figure}

\vspace{10px}\begin{definition}\label{def:guard} Guard function \\
The complete guard is created by:
\vspace{5px} \\
$\function_{\guard}:\:\function 1_{\guard} \land \function 2_{\guard}$
\end{definition}

\paragraph*{Update mapping}\label{sec:updates}
When a value edge is erased from a graph and a new value edge is created with the same label and variable anchor, this inidicates an update for the corresponding GG variable. The update mapping for the switch relation represented by this rule transition should map the matched GG variable to the interaction variable represented by the target of the new value edge.
\vspace{10px}\begin{definition}\label{def:um} Update mapping function
\vspace{2px} \\
$\function_{\updateMapping}:\: (\variableLabel, \valueLabel) \mapsto \node'\: |\: (\variableAnchor, \variableLabel, \variableAnchor) \in \DefinedHostEdges \land (\node, \valueLabel, \node') \in E_{RHS} \land \rulematch(\node) = \variableAnchor$
\end{definition}
\vspace{10px}
This poses some constraints on the eraser/creator edges for value edges, which are explained in section~\ref{sec:constraints}. These constraints ensure that $\function_{\updateMapping}$ is bijective.

\section{Rule priorities}
Rule priorities are not taken into account in the definition above. First, we show why the definition above does not work with rule priorities. Then we show a method of including rule priorities in the IOSTS.

We apply the definition to the GG in Figure~\ref{fig:priority_gg}. We assume the rules both are stimuli, i.e. each is a I/O transformation rule $(\ggrule$, $\mathit{input})$, where $\ggrule$ is the respective rule. All host graphs explored by this GG are isomorphic under the point algebra, so they represent the same location. The IOSTS obtained from this IOGG is in Figure~\ref{fig:priority_sts_wrong}, with $\imath = \{x \mapsto 25\}$. This IOSTS is wrong, because the `?sub' switch relation can be taken from the start, whereas the `?sub' rule in the IOGG cannot be applied to the start graph, because it has a lower priority than the `?add' rule.

\begin{figure}[ht]
  \begin{center}
    $\xymatrix{
   \ovalbox{$l_0$} \ar@(ul,ur)[]^{?sub\,|\,x\,>\,10\,|\,x\,:=\,x\,-\,10} \ar@(dl,dr)[]_{?add\,|\,x\,<\,30\,|\,x\,:=\,x\,+\,10}
}$

  \end{center}
  \caption{A wrong STS transformation of the graph grammar in Figure~\ref{fig:priority_gg}}
  \label{fig:priority_sts_wrong}
\end{figure}

The solution is shown in Figure~\ref{fig:priority_sts_right}. The negated guard of the `?add' switch relation is added to the `?sub' switch relation. The optimized guard for this switch relation is `x >= 30' of course, but this shows the main principle: for a rule transition $\ruleTransition$ and a set of rule transitions $\RuleTransitions$ with higher priority rules and the same source graph as $\ruleTransition$, the negated guard of the switch relations represented by $\RuleTransitions$ must be added to the guard of the switch relation represented by $\ruleTransition$. In the example, the `x < 30' guard is negated to `!(x < 30)' and added to yield the `x > 10 \&\& !(x < 30)' guard.

\begin{figure}[ht]
  \begin{center}
    $\xymatrix{
   \fbox{$l_0$} \ar@(ul,ur)[]_{sub | x > 10 \&\& !(x < 30) | x := x - 10} \ar@(dl,dr)[]_{add | x < 30 | x := x + 10}
}$

  \end{center}
  \caption{A correct STS transformation of the graph grammar in Figure~\ref{fig:priority_gg}}
  \label{fig:priority_sts_right}
\end{figure}

This problem only arises when the switch relation represented by the rule transition has a guard; if the rule has no constraints on data and if it is applicable on one graph state, it is applicable on all isomorphic graph states under the point algebra. Therefore, rule transitions with lower priority rules do not exist from that graph state and the respective switch relations also do not exist.
