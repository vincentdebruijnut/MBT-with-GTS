This section describes an algorithm for creating an STS $J: \langle \Locations_J,\initialLocation,\LocationVariables_J,\initializationFunction_J,\InteractionVariables_J,\Gates_J,\Switches_J\rangle$ from a GG $K: \langle \ruleGraph_{LHS}, \ruleGraph_{NAC}, \ruleGraph_{RHS}\rangle$ and an initial graph $\startGraph$. 

\section{Point algebra}
We define a \textit{point algebra} $\PointAlgebra$\newnot{symbol:PointAlgebra} to be an algebra with $\forall \sort \in \Sorts . |\mathbb{U}_\PointAlgebra^\sort| = 1$. The first step in the algorithm is to explore the GG $K$ using the point algebra $\PointAlgebra$ to a GTS $O_\PointAlgebra: \langle \hostGraph_O, \Rules_O, \RuleMatches_O, \RuleTransitions_O, \startGraph \rangle$.

\section{Graphs}
The graphs $\Graphs$ in a GTS are representations of the states of a transition system. In an STS, this representation is done by the locations and the location variables. In order to obtain $\Locations_J$ and $\LocationVariables_J$, the graphs $\hostGraph_K$ of $O_\PointAlgebra$ are used.

\subsection{Locations}
We define the function $\function_\Locations: \Graphs \stackrel{\sim}{\longrightarrow} \Locations$ to obtain the location from a graph. For each graph a unique location is obtained by: $\forall\: \hostGraph \in \hostGraph_O\: .\:\function_\Locations(\hostGraph) \mapsto \location \in \Locations_J$. Additionally, the initial location is obtained by: $\function_\Locations(\startGraph) \mapsto \initialLocation$.

\subsection{Location variables}
We define the function $\function_\LocationVariables: \Edges \stackrel{\sim}{\longrightarrow} \LocationVariables$ to obtain the location variable from an edge. For each edge from a graph node to a value node a unique location variable is obtained by: $\forall\: \edge \in (\DefinedEdges\restriction(\GraphNodes \times \Labels \times \mathbb{U}))\: .\: \function_\LocationVariables(\edge) \mapsto \variable \in \LocationVariables_J$. The initialization $\initializationFunction$ for a variable $\variable$, such that $\function_\LocationVariables(\node_s, \ltsLabel, \node_t) \mapsto \variable$, is obtained by: $\variable \mapsto \node_t$. Note that the node is a value, i.e. $\node_t \in \mathbb{U}$.

\section{Rule transitions}
Rule transitions are a representation of the change from one system state to another. In STSs, this representation is done by the switch relations. In order to obtain $\Switches_J$, the rules $\Rules_O$, rule matches $\RuleMatches_O$ and rule transitions $\RuleTransitions_O$ are used.

\subsection{Gates}
We define the function $\function_\Gates: \Rules \stackrel{\sim}{\longrightarrow} \Gates$ to obtain the gate from a rule. For each rule a unique gate is obtained by: $\forall\:\ggrule \in \Rules_O\: . \: \function_\Gates(\ggrule) \mapsto \gate$, where $arity(\gate) = 0$.

\subsection{Interaction variables}
The set of interaction variables for $J$ is empty, i.e. $\InteractionVariables = \emptyset$. The reason for this is that the value for these variables comes from outside the GG, namely from the SUT. In section~\ref{sec:implementation} an implementation-specific solution is described to include interaction variables in the algorithm. 

\subsection{Guards}
We define a function $\function_\guard: \RuleMatches \rightarrow \Terms$ to obtain the guard from a rule match. First, the variables in the terms in $\ruleGraph_{LHS}$ are replaced by location variabels by: $\forall (\edge:(\node, \ltsLabel, \node') \in \ruleGraph_{LHS}\restriction (\GraphNodes \times \Labels \times \Variables)\: .\: \forall \term \in 2^\Terms_{LHS}\restriction\BooleanTerms\: .\: \mathit{replace}(\node', \function_\LocationVariables(\mathit{image}(\edge)))$. The guard can then be constructed by joining the terms in $2^\Terms_{LHS}\restriction\BooleanTerms$ with the $\land$ operator.

\subsection{Update mapping}
We define a function $\function_\updateMapping: \RuleMatches \rightarrow \Terms$ to obtain the update mapping from a rule match. First we obtain all \textit{eraser-creator pairs} by: $\function_{RHS}: (\edge:(\node, \ltsLabel, \node') \in \ruleGraph_{LHS}\restriction (\GraphNodes \times \Labels \times \Variables)\: |\: \edge \notin \ruleGraph_{RHS} . (\mathit{image}(\node), \ltsLabel, \node'') \in \ruleGraph_{RHS}$. update is $\function_\LocationVariables(\edge) \mapsto \node''$, if $\node'' \in 2^\Terms_{RHS}$.

\subsection{Switch relations}
We define the function $\function_\Switches: (\hostGraph \xrightarrow{\ggrule, \rulematch} \hostGraph') \rightarrow (\function_\Locations(\hostGraph)\xrightarrow{\function_\Gates(\ggrule), \function_\guard(\rulematch), \function_\updateMapping(\rulematch)}\function_\Locations(\hostGraph'))$ to obtain a switch relation from a rule match. 

