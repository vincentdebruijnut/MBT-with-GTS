\section{Conclusions}

This report shows the usability of Graph Grammars in model-based testing. The motivation to use GGs is supported by literature, emphasizing the understandibility of graphs and the usefulness of graphs to express system states. Symbolic Transition Systems are a useful formalism which are easily used by computers. This supports using a GG for testers to model a software system and transforming the GG into an STS for the computer to use during testing. The point algebra was shown to be useful for achieving the transformation to an STS.

The link between data values in a GG and variables in an STS is harder. Viewing an edge to a data value as a variable was a choice which allowed the transformation to an STS. GGs can dynamically add variables or delete them; something which was frequently desired to obtain the smallest and most intuitive models. The constraints defined restrict this however. 

The reservation system suffered from the constraints the most. The puzzle example on the other hand shows the strength of GGs the most. This shows that some systems are more suitable to be modelled in GGs than others. 

The case study showed a real life example of a software system, succefully modelled as a GG. Here, the strength of behavioral rules is apparent: instead of state-transition thinking, the graph transformation rules allow each behavioral aspect of a software system to be modelled by one rule. This is most visible in the 'not signed on' error rule. This rule models the aspect of giving an error when a request is done in the signed off state, regardless of any other state the system is in. Restrictions of where this rule applies can be easily added, as shown with the exception made for the 'sign on' request. This shows that GGs are very useful when it comes to changing requirements: only few rules need to be adjusted to accomodate these changes. Another example is the request structure of the GG. The stimulus and expected response are automatically included for every system state. This again shows the strength of GGs; behavior is modelled once and tested in every system state. 
