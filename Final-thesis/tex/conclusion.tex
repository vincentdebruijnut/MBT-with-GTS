\section{Conclusions}\marginpar{Coverage? go back to research questions and design 'musts'}

This report shows the usability of Graph Grammars in model-based testing. The motivation to use GGs is supported by literature, emphasizing the understandibility of graphs and the usefulness of graphs to express system states. Symbolic Transition Systems are a useful formalism which are easily used by computers. This supports using a GG for testers to model a software system and transforming the GG into an STS for the computer to use during testing. The point algebra was shown to be useful for achieving the transformation to an STS.

The link between data values in a GG and variables in an STS is harder. Viewing an edge to a data value as a variable was a choice which allowed the transformation to an STS. GGs can dynamically add variables or delete them; something which was frequently desired to obtain the smallest and most intuitive models. The constraints defined restrict this however. 

The reservation system suffered from the constraints the most. The puzzle example on the other hand shows the strength of GGs the most. This shows that some systems are more suitable to be modelled in GGs than others. 
