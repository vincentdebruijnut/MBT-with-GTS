\section{Constraints}\label{sec:constraints}
In order for the definition in section~\ref{sec:algorithm} to work, four constraints are made for the IOGGs used in GRATiS. This section describes those constraints.

\paragraph*{GG variable persistency} All location variables in an STS are initialized to some value and no new variables are added. In a GG, it is possible to erase and create new variables in the transformation rules. Therefore, we pose the following constraints:
\begin{itemize}
%$\forall \langle \variableAnchor, \valueEdge, \valueLabel\rangle \in \GGVariables\:.\: \variableAnchor \in \DefinedNodes_{\startGraph} \land \valueEdge \in \DefinedEdges_{\startGraph} \land \valueEdge \in \DefinedEdges_{\startGraph}$ 
\item All GG variables must be present in the start graph
\item Variable anchors and variable edges cannot be erased or created
\item A transformation rule may erase a value edge if and only if it creates a new edge with the same source node and label as the erased value edge. The target node of the new edge must be a value node of the same sort as the target node of the erased edge
\end{itemize}

\paragraph*{Unique GG variables}
The GG variables to location variables mapping has to be bijective. The previous constraints ensure the GG variables are always present, however they do not ensure their uniqueness. Therefore, we pose the following constraints:
\begin{itemize}
\item The variable labels have to be unique; there cannot be two variable edges with the same label in the start graph
\item The value labels for the value edges with the same source node have to be unique; there cannot be two edges with the same variable anchor with the same label in the start graph
\end{itemize}

\paragraph*{No GG variables in NACs}
In graph transformation rules, it is common to create restrictions on data using a $\mathit{NAC}$. For example, the rule graph in Figure~\ref{fig:nac_example_wrong} as a $\mathit{NAC}$ expresses that the rule cannot be applied when the system is in state `done'. When using the point algebra, a problem occurs when a value edge is part of a $\mathit{NAC}$. The rule containing such a $\mathit{NAC}$ will never match any host graph, because:
\begin{enumerate}[(a)]
\item the value edge is present in every host graph as stated in the previous constraints
\item the target value node of the value edge will always be an image of the target variable node of the pre-image of the value edge. This is assuming the variable node in the $\mathit{NAC}$ is of the same sort as the value node, otherwise the $\mathit{NAC}$ would never have a match, because of the constraint on the sort of the value node of a value edge. Such a $\mathit{NAC}$ would add nothing to the rule.
\end{enumerate}

\begin{figure}[ht]
  \begin{center}
    \subfloat[GG variable in $\mathit{NAC}$]{\label{fig:nac_example_wrong}\parbox[b]{6cm}{\centering$\xymatrix{
  \fbox{system} \ar@(dl,dr)[]_{var1} \ar[d]^{status} \\
  \fbox{$``done"$}
}$
}}
    \subfloat[GG variable in $\mathit{LHS}$]{\label{fig:nac_example_right}\parbox[b]{6cm}{\centering$\xymatrix{
  \fbox{system} \ar@(dl,dr)[]_{var1} \ar[d]^{status} \\
  \fbox{$\mathit{string}\colon x$} & \fbox{$x = ``done", false$}
}$
}}
  \end{center}
  \caption{Node creating rule without structural constraint}
  \label{fig:nac_example}
\end{figure}

Therefore we pose the following constraint: A $\mathit{NAC}$ may never match the value edge and value node of a GG variable. 

The correct way of modelling the example when using the point algebra is shown as the $\mathit{LHS}$ of a rule in Figure~\ref{fig:nac_example_right}. Under the point algebra, both terms $x=``done"$ and $false$ evaluate to the same boolean value. Therefore, an image for this term node can always be found, when the term node is in the $\mathit{LHS}$ of the rule. 

The guard of the switch relation will be $(x = ``done") = false$. This leads to a case where the guard of the switch relation can never be satisfied. The following paragraph describes such a case.

\paragraph*{Data constraints in node creating rules}
A case where a guard can no longer be satisfied is shown in Figure~\ref{fig:item_example_c4}. This figure shows the $\mathit{LHS}$ and $\mathit{RHS}$ of a rule in the container-items example. The rule adds an item to the container unless it is full, i.e. has five items. If an item is added, a new node labelled `new' is created in the host graph. Under the point algebra, this rule always matches the host graph and thus this rule creates an infinite number of nodes. Therefore, the exploration never ends. This means that an IOGG with such a rule has an infinite number of graphs and the corresponding IOSTS has an infinite number of switch relations. However, the guards of the switch relations can no longer be satisfied when the location variable corresponding to the GG variable $(\mathit{var1, items})$ reaches $5$. 

The rules in an IOGG which create new nodes or edges and have constraints on data should either:
\begin{itemize}
\item erase the same amount of nodes and edges as it creates
\item have a constraint on the created node or edge, e.g. the node to create is part of the $\mathit{NAC}$ of the rule
%\item eventually not be applicable on the host graph of the IOGG and that graph should have a rule transition that returns the graph to a previously explored graph
\end{itemize}
The first constraint guarantees the exploration halts. The second constraint is only a guarantee if the node is the only element in the $\mathit{NAC}$, otherwise it is possible that the $\mathit{NAC}$ does not match the graph even with the node in it.

\begin{figure}[ht]
  \begin{center}
    \subfloat[LHS]{\label{fig:item_example_c4_lhs}\parbox[b]{6cm}{\centering$\xymatrix{
  \fbox{$\{x_1 \leq 5, \mathit{true}\}$}  \\
  \fbox{container} \ar[d]^{items} \\
  \fbox{$\mathit{int}\colon x_1$}
}$
}}\hspace{20px}
    \subfloat[RHS]{\label{fig:item_example_c4_rhs}\parbox[b]{6cm}{\centering$\xymatrix{
  \fbox{new} & \\
  \fbox{container} \ar[dr]^{items} & \\
  \fbox{$\mathit{int}\colon x_1$} & \fbox{$\{x_1 + 1\}$}
}$
}}
  \end{center}
  \caption{Node creating rule with data constraint}
  \label{fig:item_example_c4}
\end{figure}
