\section{Measurements on examples}

\subsection{Simulation and redundancy}

\paragraph*{Boardgame}
The responses used by the IOSTS and by the IOGG are different. Both models, used as examples to clarify the IOSTS and IOGG formalisms, were built with a different behavior in mind. Both allow a die to be thrown, after which the IOSTS directly moves the player to the correct location and passes the turn and the IOGG moves the player by a series of responses ended with a $!nextTurn$. Therefore, both IOSTSs do not simulate each other.

\paragraph*{Farmer-wolf-goat-cabbage puzzle}
Table~\ref{tab:fwgc-simulate} shows the results for this measurement for the puzzle. 

\begin{table}[ht]
\begin{center}
\begin{tabular}{|l|c|c|c|c|}
\hline
\textbf{Model} & \textbf{Simulates?} & \textbf{Switch relations} & \textbf{Location variables} & \textbf{Redundant?} \\ \hline
Generated IOSTS & true & 50 & 0 & false \\ \hline
Modelled IOSTS & true & 11 & 4 & false \\ \hline
\end{tabular}
\end{center}
\caption{Simulation and redundancy results for the farmer-wolf-goat-cabbage puzzle}
\label{tab:fwgc-simulate}
\end{table}

The IOGG does not use variables to track the location of each item. Therefore, the generated IOSTS has a location per state of the puzzle. Since on neither side both the number of switch relations and location variables are higher, both models are not redundant.

\paragraph*{Bar-tab system}
Table~\ref{tab:bar-tab-simulate} shows the results for this measurement for the bar-tab system. 

\begin{table}[ht]
\begin{tabular}{|l|c|c|c|c|}
\hline
\textbf{Model} & \textbf{Simulates?} & \textbf{Switch relations} & \textbf{Location variables} & \textbf{Redundant?} \\ \hline
Generated IOSTS & true & 24 & 13 & true \\ \hline
Modelled IOSTS & true & 10 & 5 & false \\ \hline
\end{tabular}
\caption{Simulation and redundancy results for the bar-tab system}
\label{tab:bar-tab-simulate}
\end{table}

The modelled IOSTS keeps the name and price of drinks as location variables, whereas the generated IOSTS hard-codes these into the guards and updates. The generated IOSTS builds a switch relation with gate $?o$ for each combination of customer and drink. It also builds a switch relation with gate $?p$ for each customer. The target locations of all these switch relations have one switch relation back to the initial location. Therefore, the number of switch relations is $3*3*2+3*1*2\: =\: 24$. This could have been avoided by modelling GG variables for the price and drinks. However, this would make the IOGG more complex. 

\paragraph*{SCRP}
Table~\ref{tab:scrp-simulate} shows the results for this measurement for the case study.

%The generated IOSTS allows every stimulus in every location. The IOSTS by hand is modelled to test a subset of the more interesting behavior.

\begin{table}[ht]
\begin{center}
\begin{tabular}{|l|c|c|c|c|}
\hline
\textbf{Model} & \textbf{Simulates?} & \textbf{Switch relations} & \textbf{Location variables} & \textbf{Redundant?} \\ \hline
Generated IOSTS & ? & 540 & 2 & ? \\ \hline
Modelled IOSTS & ? & 1302 & 2 & ? \\ \hline
\end{tabular}
\end{center}
\caption{Simulation and redundancy results for the SCRP}
\label{tab:scrp-simulate}
\end{table}

\paragraph*{Conclusions}
The farmer-wolf-goat-cabbage puzzle shows that IOGGs can often be better expressed using no variables. The bar-tab system shows that the generated IOSTS can be redundant compared to the IOSTS created by hand.\marginpar{conclusies over scrp redundancy/simulation komen nog}

\subsection{Performance}
The performance of GRATiS on all models is in Table~\ref{tab:performance}.

\begin{table}[ht]
\begin{center}
\begin{tabular}{|l|c|c|}
\hline
\textbf{Example} & \textbf{runtime (ms)} & \textbf{heap-size (MB)} \\ \hline
boardgame & 300 & 1.9 \\ \hline
farmer-wolf-goat-cabbage puzzle & 770 & 5.2 \\ \hline
bar-tab system & 250 & 2.1 \\ \hline
SCRP & 9530 & 6.6 \\ \hline
\end{tabular}
\end{center}
\caption{Performance measurements}
\label{tab:performance}
\end{table}

Comparing to the examples the case-study shows little increase in heap-size and approximately 20 times higher runtime. The algorithm scales reasonably well, considering the generated IOSTS is approximately 10 times larger in size than those of the examples.

\subsection{Model complexity}
Table~\ref{tab:halstead} shows the measurements on the operators and operands of all models.  The differences in complexity differ between the models. However, the $n_2$ and $N_1$ column jump out: the distinct number of operands is much higher for the IOGG models, but the total number of operators $N_1$ is much higher in the IOSTS models. This is based on the last three models, which express the same behavior.

\begin{table}[ht]
\begin{center}
\begin{tabular}{| l | l | c | c | c | c | c |}
  \hline
  \textbf{Example} & \textbf{Model} & $n_1$ & $n_2$ & $N_1$ & $N_2$ & Volume \\ \hline
  boardgame & IOGG & 7 & 12 & 20 & 90 & 467.27 \\
  & IOSTS & 6 & 9 & 14 & 27 & 160.18 \\ \hline
  farmer-wolf-goat-cabbage puzzle & IOGG & 3 & 12 & 31 & 190 & 863.42 \\ 
  & IOSTS & 6 & 8 & 103 & 130 & 887.11 \\ \hline
  bar-tab system & IOGG & 9 & 31 & 40 & 188 & 1213.40 \\
  & IOSTS & 8 & 15 & 66 & 134 & 904.7 \\ \hline
  SCRP & IOGG & 3 & 69 & 207 & 1506 & 10569.10 \\ 
  & IOSTS & 3 & 6 & 730 & 2594 & 10536.83 \\
  \hline
\end{tabular}
\end{center}
\caption{The Halstead measurements on the models}
\label{tab:halstead}
\end{table}

\subsection{Extendability}
The next paragraphs contain a hypothetical extension to each example. New models are given which feature the extension. In the last paragraph, the increase in model complexity for each example is given in a table.

\paragraph*{Boardgame}
The boardgame is extended to include one more player and one more location. For the IOGG, this means adding new locations and players to the initial graph. The players get a fixed order in which they play. This means that the next turn rule also has to be extended. The rest of the rules stay as they are. The extended rules are in Figure~\ref{fig:gg-bg-extended}.

\begin{figure}[ht]
  \begin{center}
    \subfloat[The initial graph]{\label{fig:start-bg-extended}% To use this figure in your LaTeX document
% import the package groove/resources/groove2tikz.sty
%
% Special colors
\begin{tikzpicture}[
% Special color styles
scale=\tikzscale]
\node[node] (n4)  at (2.200, -0.180) {\ml{\textbf{Player}\\id = 2}};
\node[node] (n12)  at (3.550, -0.615) {\ml{\textbf{Player}\\id = 3}};
\node[node] (n7)  at (2.180, -1.465) {\ml{\textbf{Location}}};
\node[node] (n11)  at (0.990, -0.600) {\ml{\textbf{Player}\\\textit{turn}\\id = 1}};
\node[node] (n8)  at (1.610, -2.765) {\ml{\textbf{Location}}};
\node[node] (n9)  at (0.930, -1.985) {\ml{\textbf{Location}}};
\node[node] (n10)  at (3.560, -2.025) {\ml{\textbf{Location}}};
\node[node] (n0)  at (4.820, -0.615) {\ml{\textbf{Die}\\rolls = 0}};
\node[node] (n6)  at (2.890, -2.760) {\ml{\textbf{Location}}};
\path[edge](n12.south -| 3.560, -2.025) -- node[lab]{at} (n10) ;
\path[edge] (n7)  -- node[lab]{next} (n10) ;
\path[edge] (n10)  -- node[lab]{next} (n6) ;
\path[edge] (n8)  -- node[lab]{next} (n9) ;
\path[edge](n11.south -| 0.930, -1.985) -- node[lab]{at} (n9) ;
\path[edge] (n9)  -- node[lab]{next} (n7) ;
\path[edge](n4.south -| 2.180, -1.465) -- node[lab]{at} (n7) ;
\path[edge] (n11)  -- node[lab]{next} (n4) ;
\path[edge] (n4)  -- node[lab]{next} (n12) ;
\path[edge](n12.west |- 0.990, -0.600) -- node[lab]{next} (n11) ;
\path[edge](n6.west |- 1.610, -2.765) -- node[lab]{next} (n8) ;
\userdefinedmacro
\end{tikzpicture}
\renewcommand{\userdefinedmacro}{\relax}
}\hspace{20px}
    \subfloat[The next turn rule]{\label{fig:nextTurn-bg-extended}% To use this figure in your LaTeX document
% import the package groove/resources/groove2tikz.sty
%
% Special colors
\begin{tikzpicture}[
% Special color styles
scale=\tikzscale]
\node[node] (n3)  at (1.075, -1.730) {\ml{\textbf{Die}\\rolls = 0}};
\node[node] (n2)  at (2.050, -0.660) {\ml{\textbf{Player}\\{\color{\green}\textit{$+$ turn}}}};
\node[node] (n0)  at (1.090, -0.650) {\ml{\textbf{Player}\\{\color{\blue}\textit{$-$ turn}}}};
\path[deledge](n0.south -| 1.075, -1.730) -- node[dellab]{throws} (n3) ;
\path[edge](n0.east |- 2.050, -0.660) -- node[lab]{next} (n2) ;
\userdefinedmacro
\end{tikzpicture}
\renewcommand{\userdefinedmacro}{\relax}
}
  \end{center}
  \caption{The extended IOGG of the board game example in Figure~\ref{fig:example_groove}}
  \label{fig:gg-bg-extended}
\end{figure}

The IOSTS gains a variable and a switch relation for the new player:
\vspace{5px} \\
$\begin{array}{lcl}
\Locations & = & \{t, m\} \\
\initialLocation & = & t \\
\LocationVariables & = & \{T, P1, P2, P3, D\} \\
\initializationFunction & = & \{T \mapsto 0, P1 \mapsto 0, P2 \mapsto 2, P3 \mapsto 0, D \mapsto 0\} \\
\InteractionVariables & = & \{d, p, l\} \\
\Gates & = & \{?throw, !move\} \\
\Switches & = & \{t\xrightarrow{?throw, 1 <= d <= 6, D \mapsto d}m, \\
 & & m\xrightarrow{!move, T=1 \land l=(P1+D)\%5, P1 \mapsto l, T \mapsto 2}t, \\
 & & m\xrightarrow{!move, T=2 \land l=(P2+D)\%5, P2 \mapsto l, T \mapsto 3}t, \\
 & & m\xrightarrow{!move, T=3 \land l=(P3+D)\%5, P3 \mapsto l, T \mapsto 1}t\}
\end{array}$

\paragraph*{Farmer-wolf-goat-cabbage puzzle}
In another variant of this puzzle, when one of the items is eaten, the puzzle can reset, but it can also undo the last action. The $\mathit{!eaten}$ rule can have either effect. In Figure~\ref{fig:gg-fwgc-extended} shows this extension for the IOGG in five rules. The rules keep track of the last moved items. When an item gets eaten, the last move can be undone.

The IOSTS in Figure~\ref{fig:sts-fwgc-extended} keeps extra variables for the previous positions of the items and adds one switch relation that restores the items to their previous positions when an item gets eaten.

\paragraph*{Bar-tab system}
The system is extended to allow ordering multiple drinks of different types. Also, a customer can purchase the option of receiving 10\% discount on all ordered drinks for 50 euros, which is added to the tab of the customer.
\vspace{5px}
\begin{itemize}
\item $?o(i,d_1,q_1,d_2,q_2,d_3,q_3)$ is used to order a quantity $q_n$ of drink $d_n$ on bar tab $i$. 
\item $?d(i)$ is used to order the discount on bar tab $i$.
\item $!pd(b)$ process the discount where $b$ is the new balance. 
\end{itemize}
\vspace{5px}
Figure~\ref{fig:gg-bartab-extended} shows the extended rules and initial graph. The $?p$ and $!pp$ rules have remained the same. The bar tabs track their discount. When processing an order, the discount is applied to the total price.

The IOSTS in Figure~\ref{fig:sts-bartab-extended} gains three location variables to track the discount for each tab. An order discount and process discount switch relations are added. Like with the IOGG, the discount is applied to the total price of the ordered drinks.

\paragraph*{SCRP}
An extended version for the IOSTS for the case study is not available, therefore the extension measurement is not done on the models for the scan-flow protocol.

\paragraph*{Model complexity increase}
Table~\ref{tab:halstead-extended} shows the measurements on the operators and operands of all extended models and the increase in model complexity. The differences in complexity differ between the models. The volume increase does not show one trend; it is much higher for the IOSTS in the farmer-wolf-goat-cabbage puzzle and much higher for the IOGG in the bar tab system.

\begin{table}[ht]
\begin{center}
\begin{tabular}{| l | l | c | c | c | c | c | c |}
  \hline
  \textbf{Example} & \textbf{Model} & $n_1$ & $n_2$ & $N_1$ & $N_2$ & Volume & Volume increase \\ \hline
  boardgame & IOGG & 6 & 12 & 20 & 105 & 521.24 & 53.97 \\
  & IOSTS & 6 & 10 & 24 & 39 & 252.0 & 91.82\\ \hline
  farmer-wolf-goat-cabbage puzzle & IOGG & 4 & 12 & 38 & 217 & 1020.00 & 156.58\\ 
  & IOSTS & 6 & 12 & 146 & 247 & 1638.78 & 751.67\\ \hline
  bar-tab system & IOGG & 9 & 36 & 65 & 290 & 1949.61 & 736.21\\
  & IOSTS & 8 & 23 & 88 & 156 & 1208.82 & 304.12\\
  \hline
\end{tabular}
\end{center}
\caption{The Halstead measurements on the extended models}
\label{tab:halstead-extended}
\end{table}

\begin{comment}
A recent extension on the protocol allows multiple accounts. While an account is not in state open, an idle account can be opened. This allows for a customer to scan his/her products, while another customer pays. Figure~\ref{fig:gg-fwgc-extended} shows the changes to the initial graph and the open account rules. Figure~\ref{fig:close-account-success-ext} shows the success response rule for closing an account: the order of closed accounts have to be kept, because the accounts have to be paid in that order.

\begin{figure}[ht]
  \begin{center}
    \subfloat[The initial graph]{\label{fig:start-scrp-ext}% To use this figure in your LaTeX document
% import the package groove/resources/groove2tikz.sty
%
% Special colors
\begin{tikzpicture}[
% Special color styles
scale=\tikzscale]
\node[node] (n0)  at (1.020, -1.065) {\ml{\textbf{CR}\\\textit{SS\_OFF}}};
\node[node] (n1)  at (2.520, -1.035) {\ml{\textbf{SFU}}};
\node[node] (n2)  at (1.025, -1.850) {\ml{\textbf{Account}\\\textit{AS\_IDLE}}};
\node[node] (n3)  at (2.055, -1.860) {\ml{\textbf{Account}\\\textit{AS\_IDLE}}};
\path[edge](n0.south -| 1.025, -1.850) -- node[lab]{has} (n2) ;
\path[edge] (n0)  -- node[lab]{has} (n3) ;
\userdefinedmacro
\end{tikzpicture}
\renewcommand{\userdefinedmacro}{\relax}
}\hspace{20px}
    \subfloat[The open account success rule]{\label{fig:open-account-success-ext}% To use this figure in your LaTeX document
% import the package groove/resources/groove2tikz.sty
%
% Special colors
\begin{tikzpicture}[
% Special color styles
scale=\tikzscale]
\node[node] (n2)  at (1.695, -0.770) {\ml{\textbf{CR}\\\textit{SS\_ON}}};
\node[node] (n0)  at (2.975, -1.975) {\ml{\textbf{Account}\\{\color{\blue}\textit{$-$ AS\_IDLE}}\\{\color{\green}\textit{$+$ AS\_OPEN}}}};
\node[node] (n11)  at (4.030, -0.775) {\ml{\textbf{SFU}}};
\node[delnode] (n3)  at (2.860, -0.770) {\ml{\textbf{Request}\\\textit{open}}};
\node[nacnode] (n1)  at (1.700, -2.030) {\ml{\textbf{Account}\\\textit{AS\_OPEN}}};
\path[edge] (n2)  -- node[lab]{has} (n0) ;
\path[nacedge](n2.south -| 1.700, -2.030) -- node[naclab]{has} (n1) ;
\path[deledge](n11.west |- 2.860, -0.770) -- node[dellab]{from} (n3) ;
\path[deledge](n3.west |- 1.695, -0.770) -- node[dellab]{to} (n2) ;
\userdefinedmacro
\end{tikzpicture}
\renewcommand{\userdefinedmacro}{\relax}
}\\
    \subfloat[The open account invalid rule]{\label{fig:open-account-invalid-ext}% To use this figure in your LaTeX document
% import the package groove/resources/groove2tikz.sty
%
% Special colors
\begin{tikzpicture}[
% Special color styles
scale=\tikzscale]
\node[node] (n2)  at (0.990, -0.875) {\ml{\textbf{CR}\\\textit{SS\_ON}}};
\node[node] (n1)  at (1.010, -1.940) {\ml{\textbf{Account}\\\textit{AS\_OPEN}}};
\node[node] (n11)  at (3.070, -0.875) {\ml{\textbf{SFU}}};
\node[delnode] (n0)  at (2.005, -0.890) {\ml{\textbf{Request}\\\textit{open}}};
\path[edge](n2.south -| 1.010, -1.940) -- node[lab]{has} (n1) ;
\path[deledge](n11.west |- 2.005, -0.890) -- node[dellab]{from} (n0) ;
\path[deledge](n0.west |- 0.990, -0.875) -- node[dellab]{to} (n2) ;
\userdefinedmacro
\end{tikzpicture}
\renewcommand{\userdefinedmacro}{\relax}
}\hspace{20px}
    \subfloat[The close account success rule]{\label{fig:close-account-success-ext}% To use this figure in your LaTeX document
% import the package groove/resources/groove2tikz.sty
%
% Special colors
\begin{tikzpicture}[
% Special color styles
scale=\tikzscale]
\node[delnode] (n8)  at (1.695, -0.760) {\ml{\textbf{Request}\\\textit{close}}};
\node[quantnode] (n4)  at (0.765, -2.515) {\ml{$\forall$}};
\node[node] (n2)  at (2.245, -2.505) {\ml{\textbf{Account}}};
\node[node] (n1)  at (0.740, -1.710) {\ml{\textbf{Account}\\\textit{AS\_CLOSED}}};
\node[node] (n0)  at (2.415, -1.705) {\ml{\textbf{Account}\\{\color{\blue}\textit{$-$ AS\_OPEN}}\\{\color{\green}\textit{$+$ AS\_CLOSED}}}};
\node[node] (n5)  at (0.595, -0.770) {\ml{\textbf{CR}\\\textit{SS\_ON}}};
\node[node] (n6)  at (2.820, -0.725) {\ml{\textbf{SFU}}};
\path[deledge](n6.west |- 1.695, -0.760) -- node[dellab]{from} (n8) ;
\path[quantedge](n1.south -| 0.765, -2.515) -- node[lab]{@} (n4) ;
\path[deledge](n8.west |- 0.595, -0.770) -- node[dellab]{to} (n5) ;
\path[nacedge] (n1)  -- node[naclab]{next} (n2) ;
\path[edge] (n5)  -- node[lab]{has} (n0) ;
\path[edge](n5.south -| 0.740, -1.710) -- node[lab]{has} (n1) ;
\path[newedge](n1.east |- 2.415, -1.705) -- node[newlab]{next} (n0) ;
\userdefinedmacro
\end{tikzpicture}
\renewcommand{\userdefinedmacro}{\relax}
}
  \end{center}
  \caption{Parts of the extended IOGG of Scanflow Cash Register Protocol}
  \label{fig:gg-fwgc-extended}
\end{figure}
\end{comment}