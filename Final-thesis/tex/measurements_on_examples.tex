\section{Measurements on examples}

\subsection{Simulation and redundancy}
Table~\ref{tab:simulation} shows the results for this measurement. It contains all examples in the `Example' column and `Model' contains `Generated' and `Modelled' to indicate which IOSTS counterpart of the example the measurements are for. The `Simulates?' column contains `Yes' if the model simulates its counterpart in the same example or `No' otherwise. `Switch relations' and `Location variables' give the number of these for each model. Using this information for both counterparts, the model is declared redundant or not in `Redundant', as described in definition~\ref{def:redundancy}. The results per example are discussed in the next paragraphs, followed by the conclusions for this measurement.

\begin{table}[ht]
\begin{center}
\begin{tabular}{|l|l|c|c|c|c|}
\hline
\textbf{Example} & \textbf{Model} & \textbf{Simulates?} & \textbf{$\Switches$} & \textbf{$\LocationVariables$} & \textbf{Redundant?} \\ \hline
Boardgame & Generated & No & 12 & 3 & - \\ \cline{2-6}
 & Modelled & No & 3 & 4 & - \\ \hline
FWGC & Generated & Yes & 50 & 0 & No \\ \cline{2-6}
 & Modelled & Yes & 11 & 4 & No \\ \hline
Bar tab & Generated IOSTS & Yes & 24 & 13 & Yes \\ \cline{2-6}
 & Modelled IOSTS & Yes & 10 & 5 & No \\ \hline
Reservations & Generated IOSTS & - & - & - & - \\ \cline{2-6}
 & Modelled IOSTS & - & - & - & - \\ \hline
SCRP & Generated IOSTS & Yes & 540 & 2 & No \\ \cline{2-6}
 & Modelled IOSTS & No & 1302 & 2 & Yes \\ \hline
\end{tabular}
\end{center}
\caption{Simulation and redundancy results}
\label{tab:simulation}
\end{table}

\paragraph*{Boardgame}
The responses used by the IOSTS and by the IOGG are different. Both models, used as examples to clarify the IOSTS and IOGG formalisms, were built with a different behavior in mind. Both allow a die to be thrown, after which the IOSTS directly moves the player to the correct location and passes the turn and the IOGG moves the player by a series of responses ended with a $!nextTurn$. Therefore, both IOSTSs do not simulate each other.

\paragraph*{FWGC}
The IOGG does not use variables to track the location of each item. Therefore, the generated IOSTS has a location per state of the puzzle. Since on neither side both the number of switch relations and location variables are higher, both models are not redundant.

\paragraph*{Bar-tab system}
The modelled IOSTS keeps the name and price of drinks as location variables, whereas the generated IOSTS hard-codes these into the guards and updates. The generated IOSTS builds a switch relation with gate $?o$ for each combination of customer and drink. It also builds a switch relation with gate $?p$ for each customer. The target locations of all these switch relations have one switch relation back to the initial location. Therefore, the number of switch relations is $3*3*2+3*1*2\: =\: 24$. This could have been avoided by modelling GG variables for the price and drinks. However, this would make the IOGG more complex. 

\paragraph*{SCRP}
The generated IOSTS allows every stimulus in every location. The modelled IOSTS is modelled to test a subset of the more interesting behavior. Therefore, the generated IOSTS simulates the modelled IOSTS, but not vice versa. The modelled IOSTS is also redundant, because it features many $\tau$ transitions.

\paragraph*{Conclusions}
This measurement shows that IOGGs are good at making a model input-complete. For the smaller examples, it shows that the modelled IOSTSs often have fewer locations than the generated IOSTSs. Presumably, when modelling with IOGGs, making rules that create new locations are preferred over adding GG variables, if possible.

\subsection{Model complexity}
Table~\ref{tab:halstead} shows the measurements on the operators and operands of all models.  The differences in complexity differ between the models. However, the $n_2$ and $N_1$ column jump out: the distinct number of operands is much higher for the IOGG models, but the total number of operators $N_1$ is much higher in the IOSTS models. This is based on the last three models, which express the same behavior.

\begin{table}[ht]
\begin{center}
\begin{tabular}{| l | l | c | c | c | c | c |}
  \hline
  \textbf{Example} & \textbf{Model} & $n_1$ & $n_2$ & $N_1$ & $N_2$ & Volume \\ \hline
  Boardgame & IOGG & 7 & 12 & 20 & 90 & 467.27 \\ \cline{2-7}
  & IOSTS & 6 & 9 & 14 & 27 & 160.18 \\ \hline
  FWGC & IOGG & 3 & 12 & 31 & 190 & 863.42 \\ \cline{2-7}
  & IOSTS & 6 & 8 & 103 & 130 & 887.11 \\ \hline
  Bar tab & IOGG & 9 & 31 & 40 & 188 & 1213.40 \\ \cline{2-7}
  & IOSTS & 8 & 15 & 66 & 134 & 904.7 \\ \hline
  Reservations & IOGG & 1 & 19 & 5 & 50 & 237.71 \\ \cline{2-7}
  & IOSTS & - & - & - & - & - \\ \hline
  SCRP & IOGG & 3 & 69 & 207 & 1506 & 10569.10 \\ \cline{2-7}
  & IOSTS & 3 & 6 & 730 & 2594 & 10536.83 \\ \hline
\end{tabular}
\end{center}
\caption{The Halstead measurements on the models}
\label{tab:halstead}
\end{table}

\subsection{Extendability}
The next paragraphs contain a hypothetical extension to each example. New models are given which feature the extension. In the last paragraph, the increase in model complexity for each example is given in a table. The restaurant reservation system is omitted from this measurement, as is SCRP.

\paragraph*{Boardgame}
The boardgame is extended to include one more player and one more location. For the IOGG, this means adding new locations and players to the initial graph. The players get a fixed order in which they play. This means that the next turn rule also has to be extended. The rest of the rules stay as they are. The extended rules are in Figure~\ref{fig:gg-bg-extended}.

\begin{figure}[ht]
  \begin{center}
    \subfloat[The initial graph]{\label{fig:start-bg-extended}% To use this figure in your LaTeX document
% import the package groove/resources/groove2tikz.sty
%
% Special colors
\begin{tikzpicture}[
% Special color styles
scale=\tikzscale]
\node[node] (n4)  at (2.200, -0.180) {\ml{\textbf{Player}\\id = 2}};
\node[node] (n12)  at (3.550, -0.615) {\ml{\textbf{Player}\\id = 3}};
\node[node] (n7)  at (2.180, -1.465) {\ml{\textbf{Location}}};
\node[node] (n11)  at (0.990, -0.600) {\ml{\textbf{Player}\\\textit{turn}\\id = 1}};
\node[node] (n8)  at (1.610, -2.765) {\ml{\textbf{Location}}};
\node[node] (n9)  at (0.930, -1.985) {\ml{\textbf{Location}}};
\node[node] (n10)  at (3.560, -2.025) {\ml{\textbf{Location}}};
\node[node] (n0)  at (4.820, -0.615) {\ml{\textbf{Die}\\rolls = 0}};
\node[node] (n6)  at (2.890, -2.760) {\ml{\textbf{Location}}};
\path[edge](n12.south -| 3.560, -2.025) -- node[lab]{at} (n10) ;
\path[edge] (n7)  -- node[lab]{next} (n10) ;
\path[edge] (n10)  -- node[lab]{next} (n6) ;
\path[edge] (n8)  -- node[lab]{next} (n9) ;
\path[edge](n11.south -| 0.930, -1.985) -- node[lab]{at} (n9) ;
\path[edge] (n9)  -- node[lab]{next} (n7) ;
\path[edge](n4.south -| 2.180, -1.465) -- node[lab]{at} (n7) ;
\path[edge] (n11)  -- node[lab]{next} (n4) ;
\path[edge] (n4)  -- node[lab]{next} (n12) ;
\path[edge](n12.west |- 0.990, -0.600) -- node[lab]{next} (n11) ;
\path[edge](n6.west |- 1.610, -2.765) -- node[lab]{next} (n8) ;
\userdefinedmacro
\end{tikzpicture}
\renewcommand{\userdefinedmacro}{\relax}
}\hspace{20px}
    \subfloat[The next turn rule]{\label{fig:nextTurn-bg-extended}% To use this figure in your LaTeX document
% import the package groove/resources/groove2tikz.sty
%
% Special colors
\begin{tikzpicture}[
% Special color styles
scale=\tikzscale]
\node[node] (n3)  at (1.075, -1.730) {\ml{\textbf{Die}\\rolls = 0}};
\node[node] (n2)  at (2.050, -0.660) {\ml{\textbf{Player}\\{\color{\green}\textit{$+$ turn}}}};
\node[node] (n0)  at (1.090, -0.650) {\ml{\textbf{Player}\\{\color{\blue}\textit{$-$ turn}}}};
\path[deledge](n0.south -| 1.075, -1.730) -- node[dellab]{throws} (n3) ;
\path[edge](n0.east |- 2.050, -0.660) -- node[lab]{next} (n2) ;
\userdefinedmacro
\end{tikzpicture}
\renewcommand{\userdefinedmacro}{\relax}
}
  \end{center}
  \caption{The extended IOGG of the board game example in Figure~\ref{fig:example_groove}}
  \label{fig:gg-bg-extended}
\end{figure}

The IOSTS gains a variable and a switch relation for the new player:
\vspace{5px} \\
$\begin{array}{lcl}
\Locations & = & \{t, m\} \\
\initialLocation & = & t \\
\LocationVariables & = & \{T, P1, P2, P3, D\} \\
\initializationFunction & = & \{T \mapsto 0, P1 \mapsto 0, P2 \mapsto 2, P3 \mapsto 0, D \mapsto 0\} \\
\InteractionVariables & = & \{d, p, l\} \\
\Gates & = & \{?throw, !move\} \\
\Switches & = & \{t\xrightarrow{?throw, 1 <= d <= 6, D \mapsto d}m, \\
 & & m\xrightarrow{!move, T=1 \land l=(P1+D)\%5, P1 \mapsto l, T \mapsto 2}t, \\
 & & m\xrightarrow{!move, T=2 \land l=(P2+D)\%5, P2 \mapsto l, T \mapsto 3}t, \\
 & & m\xrightarrow{!move, T=3 \land l=(P3+D)\%5, P3 \mapsto l, T \mapsto 1}t\}
\end{array}$

\paragraph*{FWGC}
In another variant of this puzzle, when one of the items is eaten, the puzzle can reset, but it can also undo the last action. The $\mathit{!eaten}$ rule can have either effect. In Figure~\ref{fig:gg-fwgc-extended} shows this extension for the IOGG in five rules. The rules keep track of the last moved items. When an item gets eaten, the last move can be undone.

The IOSTS in Figure~\ref{fig:sts-fwgc-extended} keeps extra variables for the previous positions of the items and adds one switch relation that restores the items to their previous positions when an item gets eaten.

\paragraph*{Bar tab system}
The system is extended to allow ordering multiple drinks of different types. Also, a customer can purchase the option of receiving 10\% discount on all ordered drinks for 50 euros, which is added to the tab of the customer.
\vspace{5px}
\begin{itemize}
\item $?o(i,d_1,q_1,d_2,q_2,d_3,q_3)$ is used to order a quantity $q_n$ of drink $d_n$ on bar tab $i$. 
\item $?d(i)$ is used to order the discount on bar tab $i$.
\item $!pd(b)$ process the discount where $b$ is the new balance. 
\end{itemize}
\vspace{5px}
Figure~\ref{fig:gg-bartab-extended} shows the extended rules and initial graph. The $?p$ and $!pp$ rules have remained the same. The bar tabs track their discount. When processing an order, the discount is applied to the total price.

The IOSTS in Figure~\ref{fig:sts-bartab-extended} gains three location variables to track the discount for each tab. An order discount and process discount switch relations are added. Like with the IOGG, the discount is applied to the total price of the ordered drinks.

\paragraph*{SCRP}
An extended version for the IOSTS of SCRP by Axini is not available as there is no extended version of the SUT. Creating an extended IOSTS for SCRP and a corresponding SUT is out of scope for this report. Therefore the extension measurement is not done on the models for SCRP.

\paragraph*{Model complexity increase}
Table~\ref{tab:halstead-extended} shows the measurements on the operators and operands of all extended models and the increase in model complexity. The differences in complexity differ between the models. The volume increase does not show one trend; it is much higher for the IOSTS in the farmer-wolf-goat-cabbage puzzle and much higher for the IOGG in the bar tab system.

\begin{table}[ht]
\begin{center}
\begin{tabular}{| l | l | c | c | c | c | c | c |}
  \hline
  \textbf{Example} & \textbf{Model} & $n_1$ & $n_2$ & $N_1$ & $N_2$ & Volume & Volume increase \\ \hline
  boardgame & IOGG & 6 & 12 & 20 & 105 & 521.24 & 53.97 \\ \cline{2-7}
  & IOSTS & 6 & 10 & 24 & 39 & 252.0 & 91.82\\ \hline
  FWGC & IOGG & 4 & 12 & 38 & 217 & 1020.00 & 156.58 \\ \cline{2-7}
  & IOSTS & 6 & 12 & 146 & 247 & 1638.78 & 751.67\\ \hline
  bar-tab system & IOGG & 9 & 36 & 65 & 290 & 1949.61 & 736.21 \\ \cline{2-7}
  & IOSTS & 8 & 23 & 88 & 156 & 1208.82 & 304.12 \\ \hline
  Reservations & IOGG & - & - & - & - & - \\ \cline{2-7}
  & IOSTS & - & - & - & - & - \\ \hline
\end{tabular}
\end{center}
\caption{The Halstead measurements on the extended models}
\label{tab:halstead-extended}
\end{table}

\subsection{Performance}
The performance of GRATiS on all models is in Table~\ref{tab:performance}.

\begin{table}[ht]
\begin{center}
\begin{tabular}{|l|c|c|}
\hline
\textbf{Example} & \textbf{runtime (ms)} & \textbf{heap-size (MB)} \\ \hline
Boardgame & 300 & 1.9 \\ \hline
FWGC & 770 & 5.2 \\ \hline
Bar tab & 250 & 2.1 \\ \hline
Reservations & - & - \\ \hline
SCRP & 9530 & 6.6 \\ \hline
\end{tabular}
\end{center}
\caption{Performance measurements}
\label{tab:performance}
\end{table}

The algorithm scales reasonably well; the IOGG of SCRP is on average 13 times more complex than the examples and the runtime is on average 23 times longer. The space usage also shows very little increase.

\subsection{Measurement conclusions}
Table~\ref{tab:measurement_conclusions} shows the results of the measurements. The generated IOSTS in the bar tab example is redundant compared to the modelled IOSTS, but the modelled IOSTS in the SCRP case study is redundant compared to the generated IOSTS. The other examples do not show any redundancy. Therefore it is inconclusive whether the generated IOSTS is often redundant. More case studies have to be done to draw this conclusion.

\begin{table}[ht]
\begin{center}
\begin{tabular}{| l | c | c | c |}
  \hline
  \textbf{Model} & \textbf{Simulation and Redundancy} & \textbf{Model complexity} & \textbf{Extendability} \\ \hline
  IOGG & +/- & +/- & +/-\\ \hline
  IOSTS & +/- & +/- & +/-\\ \hline
\end{tabular}
\end{center}
\caption{Measurement conclusions}
\label{tab:measurement_conclusions}
\end{table}

Both the model complexity and extendability measurements are inconclusive. However, it is revealed that IOGGs use more operators and IOSTSs use more operands to express the same behavior. More larger case studies have to done to draw a conclusion. However, judging from the effort and time put into modelling and extending the models from both formalisms, the IOGGs were easier to model, because less operands were needed. The IOGG formalism in itself is more complex than the IOSTS, because it has more operators. This makes IOGGs harder to understand initially, but takes less effort to work with when a tester is adept with graph transformations. The Halstead measurement assumes more faults are generated when the code-base becomes larger and more complex. A problem indicated in the introduction is that stakeholders should validate the models for correctness. A large IOSTS is difficult to check and a person not adept in graph transformation will have an equally hard time checking an IOGG. However, when the stakeholder becomes more adept at graph transformations, the IOGGs will require less effort to check than the IOSTSs. In conclusion, the Halstead measurement is possibly not the right measurement to indicate whether a certain formalism is more usable than the other.