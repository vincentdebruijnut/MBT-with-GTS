\section{Graph Transformation}\label{sec:graph}
A \textit{graph grammar} is composed of a start graph and a set of transformation rules. The start graph describes the system in its initial state. The transformation rules describe what changes are made to the graph, resulting in a new graph which describes the system in its new state. The definition is as follows.

\begin{definition}
A graph grammar is a tuple $\langle G, R\rangle$, where:
\begin{itemize}
  \item $G$ is the start graph
  \item $R$ is a set of graph transformation rules
\end{itemize}
\end{definition} 

The rest of this section is ordered as follows: first, graphs and graph morphisms are explained. This is then used to explain graph transformation rules, followed by the definition of a graph grammar. Then, the definition of a \textit{Graph Transition System} (GTS) is given. An example of a graph grammar and a GTS is then given. Finally, a method for transforming a GTS to an STS is given. For a more detailed overview of graph grammars, we refer to~\cite{Rensink:graph_grammars, Heckel2006187, Andries1999}.

\subsection{Graphs \& morphisms}
\begin{definition}
A graph is a tuple $\langle L, N, E\rangle$, where:
\begin{itemize}
  \item $L$ is a set of labels
  \item $N$ is a set of nodes, where each $n \in N$ has a label $l \in L$
  \item $E$ is a set of edges, where each $e \in E$ has a label $l \in L$ and nodes $source,target \in N$
\end{itemize}
\end{definition}

A graph $H$ has an \textit{occurrence} in a graph $G$, denoted by $H \rightarrow G$, if there is a mapping from the nodes and the edges of $H$ to the nodes and the edges of $G$ respectively. Such a mapping is called a \textit{morphism}. An element $e$ in graph $H$ is then said to have an \textit{image} in graph $G$ and $e$ is a \textit{pre-image} of the image. A graph $H$ has a partial morphism to a graph $G$ if there are elements in $H$ without an image in $G$.

\subsection{Graph transformation rules}
\begin{definition}
A transformation rule is a tuple $\langle \mathit{LHS}, \mathit{NAC}, \mathit{RHS}, \mathit{M}\rangle$, where:
\begin{itemize}
  \item $\mathit{LHS}$ is a graph representing the left-hand side of the rule
  \item $\mathit{NAC}$ is a set of graphs representing the negative application conditions
  \item $\mathit{RHS}$ is a graph representing the right-hand side of the rule
  \item $\mathit{M_{RHS}}$ is a partial morphism of $\mathit{LHS}$ to $\mathit{RHS}$ 
  \item $\mathit{M_{NAC}}$ are partial morphisms of $\mathit{LHS}$ to each $n \in \mathit{NAC}$
\end{itemize}
\end{definition}

A rule $R$ is applicable on a graph $G$ if its $\mathit{LHS}$ has an occurrence in $G$ and $\not\ exists n \in \mathit{NAC}$ such that $n$ has an occurence in $G$ and $\forall e \in \mathit{LHS}$, if $e$ has an image $i$ in $n$ and an image $j$ in $G$, then $j$ should be an image of $j$. After the rule application, all elements in $\mathit{LHS}$ not part of $\mathit{M_{RHS}}$, i.e. they do not have an image in $\mathit{RHS}$, are removed from $G$ and all elements in $\mathit{RHS}$ not part of $\mathit{M_{RHS}}$, i.e. they do not have a pre-image in $\mathit{LHS}$, are added to $G$.

\subsection{Graph Transition Systems}
By repeatedly applying graph transformation rules to the start graph and all its consecutive graphs, a graph grammar can be explored to reveal a \textit{Graph Transition System} (GTS). This transition system consists of \textit{graph states} connected by \textit{rule transitions}.

\begin{definition}
A graph transition system is an 8-tuple	$\langle S, L, T, G, R, M_G, M_R, s0\rangle$, where:
\begin{itemize}
\item $S$ is a finite, non-empty set of graph states
\item $L$ is a finite set of labels. For each label $l\in L$, the arity of $l$, denoted $\mathit{arity(l)}$, is a natural number. The parameters of $l$, denoted $\mathit{param(l)}$, is a tuple of length $\mathit{arity(l)}$ of variables.
\item $T \in S \times (L \cup \{\tau\}) \times S$, with $\tau \notin L$, is the rule transition relation. The parameters of a transition $t \in T$ with label $l \in L$, denoted $\mathit{param(t)}$, is a tuple of length $\mathit{arity(l)}$ of constants, such that $\Sigma:\mathit{param(l)} \mapsto \mathit{param(t)}$ is the valuation of the variables of the label.
\item $G$ is a set of graphs.
\item $R$ is a set of rules.
\item $M_G$ is a mapping $\forall s \in S . s \mapsto g \in G \land \not\exists s' \in S . s \neq s' \land s' \mapsto g \in M_G$
\item $M_R$ is a mapping $\forall t \in T . t \mapsto r \in R \land \not\exists t' \in T . t \neq t' \land t' \mapsto r \in M_R$
\item $s0 \in s$ is the initial graph state.
\end{itemize}
We write $s \xrightarrow{\mu}s'$ if there is a rule transition labelled $\mu$ from state s to state s', i.e., $(s, \mu, s') \in T$.
\end{definition}

These systems are very similar to LTSs. A GTS can be transformed to an LTS by omitting the graphs, rules, mappings and parameters on labels.

\subsection{Example}\label{sec:gts_example}
Figure \ref{fig:gts} shows an example of the start graph and one rule of a graph grammar. $\mathit{M_{RHS}}$ maps the $A$ and $B$ nodes in $\mathit{LHS}$ to the A and B nodes in $\mathit{RHS}$ respectively. $\mathit{M_{NAC}}$ maps the $A$ node in $\mathit{LHS}$ to the $A$ node in both graphs in $\mathit{NAC}$. The a-edge in $\mathit{LHS}$ is mapped to the a-edge in the first $\mathit{NAC}$. The $\mathit{LHS}$ of the rule has an occurrence in the start graph, as the $A$ and $B$ nodes connected by the a-edge exist in both graphs. None of the graphs in the $\mathit{NAC}$ have an occurrence in the start graph, because the $C$ node does not exist in the start graph. The new graph after applying the rule is in Figure~\ref{fig:gg_result}.

\begin{figure}[ht]
  \begin{center}
    \subfloat[The start graph]{\label{fig:gg_graph}$\xymatrix{
   \fbox{A} \ar[r]^{a} & \fbox{B}
}$
}\hspace{20px}
    \subfloat[The LHS]{\label{fig:gg_lhs}$\xymatrix{
   \fbox{A} \ar[r]^{a} & \fbox{B}
}$
}\hspace{20px}
    \subfloat[The first NAC]{\label{fig:gg_nac1}$\xymatrix{
   \fbox{A} \ar[r]^{a} & \fbox{C}
}$
}\hspace{20px}
    
    \subfloat[The second NAC]{\label{fig:gg_nac2}$\xymatrix{
   \bullet \ar@(dl,dr)[]_{A} \ar[r]^{b} & \bullet \ar@(dl,dr)[]_{C}
}$
}\hspace{20px}
    \subfloat[The RHS]{\label{fig:gg_rhs}$\xymatrix{
   \bullet \ar@(dl,dr)[]_{A} \ar[r]^{b} & \bullet \ar@(dl,dr)[]_{B}
}$
}\hspace{20px}
    \subfloat[The result]{\label{fig:gg_result}$\xymatrix{
   \bullet \ar@(dl,dr)[]_{A} \ar[r]^{b} & \bullet \ar@(dl,dr)[]_{B}
}$
}
  \end{center}
  \caption{An example of a graph grammar}
  \label{fig:gts}
\end{figure}

\subsection{GTS to STS transformation}\label{sec:gts_sts_trafo}
The method described here transforms a GTS to an STS. This STS is not \textit{optimal}, i.e. it is reducible to an STS with fewer locations and switch relations. This method only serves as a proof used in section~\ref{sec:init_design}.

For each graph state $s \in S$ create a location $l \in L$. The start location $l_0 \in L$ is the location created by the state $s0 \in S$. Set $\mathcal{V} = \emptyset$ and $\imath = \emptyset$. For each label $l \in L$ create a gate $\lambda \in \Lambda$. For each $p \in \mathit{param}(l)$ create an interaction variable $i \in \mathcal{I}$ of the same type as $p$ and add $i$ to $param(\lambda)$. Set $\mathit{arity}(\lambda) = \mathit{arity}(l)$. For each rule transition $t \in T$ create a switch relation $r \in \rightarrow$, where:
\begin{itemize}
  \item the source and the target locations of $r$ are the locations created by the source and target graph states of the rule transition.
  \item the gate of $r$ is the gate created by the label on the transition.
  \item for each $p \in \mathit{param}(l)$ and $i \in \mathcal{I}$ created by $p$ create an $i = p$ expression. The guard of $r$ is those expressions joined by the $\land$ operator.
  \item the update mapping of $r$ is empty.
\end{itemize}

An example of such a transformation is done using the LTS in Figure~\ref{fig:trafo_lts}. Consider this LTS to be a GTS where the states are graph states, the transitions are rule transitions with label 'do' with $\mathit{arity(do)} = 1$ and $\mathit{param(do)}$ the integers between brackets are The STS resulting from the transformation is in Figure~\ref{fig:trafo_basic_sts}.

\begin{figure}[ht]
  \begin{center}
    $\xymatrix{
   \bullet \ar[rr]^{do(n:N) | n = 1 | } \ar[dd]_{do(n:N) | n = 2 | } && \bullet \ar@(ul,ur)^{do(n:N) | n = 1 | } \ar@(d,r)[ddll]^{do(n:N) | n = 2 | } \\ \\
   \bullet \ar@/^/[uurr]|-{do(n:N) | n = 1 | } \ar@(dl,dr)[]_{do(n:N) | n = 2 | }}$
  \end{center}
  \caption{The basic STS resulting from the transformation of the LTS in Figure~\ref{fig:trafo_lts}}
  \label{fig:trafo_basic_sts}
\end{figure}
