This chapter covers the design issues of GRATiS. GRATiS transforms a graph grammar to an STS and starts the model-based testing on that STS. This design choice was made, because STSs are practical for model-based testing; variables give support for modelling data values in systems and using STSs supports the separation of state/transition coverage in location/switch relation coverage and data coverage.

Section~\ref{sec:gg-to-sts} gives a formal approach of transforming a graph grammar to an STS. Section \ref{sec:gg-exploration} gives possible optimizations. Section~\ref{sec:gratis-design} shows the design of GRATiS and elaborates on the choices made. Section~\ref{sec:implementation} gives implementation problems and solutions on GROOVE and ATM.
