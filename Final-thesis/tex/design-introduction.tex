This chapter covers the design issues of GRATiS. The concrete method chosen to construct GRATiS is to transform a graph grammar into an STS in GROOVE and do the model-based testing with that STS in ATM\marginpar{revise this sentence}. Why? STSs are practical for model-based testing, variables give greater support for modelling data values in systems, extra coverage statistics over locations and switch relations which represent system state without specifying data coverage.

Section~\ref{sec:gg-to-sts} gives a formal approach to transforming a graph grammar into an STS. Sections \ref{sec:exploration} through \ref{sec:reachability} each cover a specific design problem with possible solutions. Section~\ref{sec:pre-testing-vs-on-the-fly} gives two possible implementations of GRATiS with the advantages and disadvantages of each implementation.
