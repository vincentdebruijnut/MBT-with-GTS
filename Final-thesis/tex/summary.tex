\section{Summary}\label{sec:conclusion}
This report motivates the need of a research towards a model-based testing practice on Graph Transformation Systems. The goal will be to create a tool that allows automatic test generation on GTSs and assess the strengths and weaknesses of this test practice.

The first observations in this report demonstrate that GTSs can provide a nice visiualization of a system. However, the representation of data values and in a GTS were not   Increasing the complexity of the software system may change this. The transformation rules, given as separate graphs, provide a good overview of the behavior of the system. This feature should be beneficial in models for larger software systems.

The results also show an interesting automatic statespace reduction, namely the symmetry reduction. The second GTS of the example shows that also a purely arithmetic model can be built; this indicates the strength of the formalism.

The design phase is split into small steps that should break down the complexity of the entire implementation process. The validation with the use of the case studies emphasises the practicality of the tooling; the purpose of the test tools is to be used on real-world software systems. Finally, the experiments are a great indication of the usefulness of the tool towards software testers.