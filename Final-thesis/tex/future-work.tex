\section{Future work}\label{sec:future-work}
Better measurements to evaluate model complexity, such as sociological experiments, can be used to improve the usefulness of GGs in model-based testing. For instance, these measurements can be used to improve tools such as GROOVE, by making the GGs easier to create and understand.

When STSs are extended to allow more complex data structures, such as sets, maps and arrays, the definition of GG variables can also be extended. This should allow testers to viably model IOGGs for more software systems. Alternatively, other strengths of GGs can be explored, such as combining the consecutive application of the same rule into one, as was shown in the boardgame example.

More case studies on real-life software systems should be done, in order to establish and improve the usefulness of GRATiS to the model-based testing paradigm.