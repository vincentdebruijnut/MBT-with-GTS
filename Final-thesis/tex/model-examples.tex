\section{Validation models}\label{sec:model-examples}

Several examples of small systems are used to validate GRATiS. Each example is modelled using a GG and an STS. The examples are:
\begin{itemize}
\item a boardgame - a game with a stochastic element
\item the farmer-wolf-goat-cabbage puzzle - a datafree puzzle with states
\item a reservation system - a system where people can make reservations for tables at a restaurant
\item a bar tab - a system where people order drinks on their tab
\begin{comment}
\item a communication protocol
\end{comment}
\end{itemize}

In addition, a case study is done where GRATiS is tested on a larger software system. The system is a \textit{self-scan register}, which allows customers of a supermarket to scan and pay for their products without help of an employee. The system contains a \textit{scan-flow unit}, which scans the products, and a \textit{cash register unit}, which allows for the payment. The communication protocol between these two units is modelled as a GG, based on the specification of the system. This model is used for the model-based testing against the live system.
