\section{Symbolic Graph Grammar Exploration}\label{sec:gg-exploration}

In the previous section, rules are given for transforming graph states and rule transitions to an STS. However, completely exploring the GTS is often unfeasable, as GTSs, like LTSs, have states for each concrete data value. This section describes an exploration technique to transform a graph grammar to an STS without having to explore the entire GTS.

\subsection{Partial matching}\label{sec:partial-matching}
When a rule match of a rule $r$ exists on a graph $G$, if the $\mathit{LHS}$ of $r$ is allowed to only have a partial morphism in a graph $G$, this is called a \textit{partial match} on $G$. Note that according to the definition given in section~\ref{sec:graph_rules}, $r$ is applicable on $G$ when a $\mathit{NAC}$ has no occurrence in $G$, where an image of an element in the $\mathit{NAC}$ should also be an image of an element in the $\mathit{LHS}$. This means that a rule which does not have a match on a graph, because of its $\mathit{NAC}$, may still have a partial match on the graph. For example, the $\mathit{LHS}$ in Figure~\ref{fig:lhs_trafo} has a partial morphism on the graph in Figure~\ref{fig:variable_node}, if the value in the latter is '10' instead of '25'. The rule with this $\mathit{LHS}$ and the $\mathit{NAC}$ in Figure~\ref{fig:nac_trafo} has a partial match on this graph, as the edge labelled '10' has no image in the $\mathit{LHS}$.

When a rule has a partial match, excluding the value nodes, on the graph of a graph state $s$, it indicates the existence of an outgoing switch relation from the location represented by the abstraction of $s$ in the STS transformation. The guard and update mapping can be constructed from the rule, by inspecting the variable and value nodes in the $\mathit{LHS}$, $\mathit{NAC}$ and $\mathit{RHS}$. Whether this switch relation is ever enabled, depends on whether the graph states on which the rule has a full match are reachable from the start graph state. When exploring a GTS with this partial matching technique, graph states can be omitted from the exploration when their abstraction has already been encountered. This ensures that for each switch relation in the STS transformation, only one rule transition is explored.

The rule in Figure~\ref{fig:algebra} in section~\ref{sec:variables-in-graph-grammars} leads to a new graph state for each possible value of 'x'. A graph grammar with this rule and the initial graph state in Figure~\ref{fig:algebra_start_graph} leads to a GTS as shown in Figure~\ref{fig:algebra_gts}, where $s_0$, $s_1$ and $s_2$ are representations of the start graph state, the graph state after one rule application and  the graph state after two rule applications respectively. The abstractions of all three graph states are isomorphic, because only the value of the variable 'x' changes. The transformation of this graph grammar to an STS using partial matching is as follows: The start graph state is transformed to the start location $l_0$. The value nodes are not included in the partial match, so the rule transition leads to the same graph state. This partial rule match is transformed to a switch relation with gate $add$ from $l_0$ to $l_0$. By inspecting the LHS, the guard on $r$ can be set to $x <= y$. By inspecting the RHS, the update mapping on $r$ can be set to $x \mapsto x + y$. The result is shown in Figure~\ref{fig:algebra_sts}.

\begin{figure}[ht]
  \begin{center}
    $\xymatrix{
   \bullet \ar@(dl,dr)[]_{x} \ar[r]^{value} & \bullet \ar@(dl,dr)[]_{0} \\
   \bullet \ar@(dl,dr)[]_{y} \ar[r]^{value} & \bullet \ar@(dl,dr)[]_{5}
}$

  \end{center}
  \caption{A start graph state}
  \label{fig:algebra_start_graph}
\end{figure}

\begin{figure}[ht]
  \begin{center}
    $\xymatrix{
  \fbox{$s_0$} \ar[r]^{add(0,5)} & \fbox{$s_1$} \ar[r]^{add(5,5)} & \fbox{$s_2$}
}$

  \end{center}
  \caption{The GTS resulting from Figures \ref{fig:algebra_start_graph} and \ref{fig:algebra}}
  \label{fig:algebra_gts}
\end{figure}

\begin{figure}[ht]
  \begin{center}
    $\xymatrix{
  \fbox{$l_0$} \ar@(dl,dr)[]_{add\,|\,x\,<=\,y\,|\,x\,:=\,x\,+\,y}
 }$

  \end{center}
  \caption{The result of the graph grammar to STS transformation}
  \label{fig:algebra_sts}
\end{figure}

This exploration technique with partial matching can potentially lead to an infinitely continuing exploration. An example of this is given using the same start graph and rule as in Figures~\ref{fig:algebra_start_graph} and \ref{fig:algebra}. However, the RHS of the rule is changed to the one in Figure~\ref{fig:algebra_rhs2}. A new graph state is found with each application of the rule. With the partial matching system, this means that the STS from the transformation will expand infinitely. With the normal matching system, this is not the case, as the rule can only be applied two times and then the rule does not match anymore. One solution is to set a modelling constraint on GRATiS. This constraint disallows the use of a rule that causes an infinite expansion of graph states with the partial matching technique. Another solution is to allow user input for the maximum depth of the exploration. However, the coverage statistics for such models are often incorrect, as many locations are still unreachable. Section~\ref{sec:reachability} provides a solution to this problem. 

\begin{figure}[ht]
  \begin{center}
    $\xymatrix{
   \bullet \ar@(dl,dr)[]_{x} \ar[r]^{value} & \bullet \ar[d]^{+} \\
   \bullet \ar@(dl,dr)[]_{y} \ar[r]^{value} & \bullet \\
   \bullet \ar@(dl,dr)[]_{new}
}$

  \end{center}
  \caption{The RHS of a rule leading to infinite continuing exploration}
  \label{fig:algebra_rhs2}
\end{figure}

\subsection{Reachability}\label{sec:reachability}
With each new location in the transformed STS the question arises whether that location is reachable from the start location. The MSc thesis of Sietsma~\cite{Sietsma:reachability} gives an algorithm for checking whether a location is reachable. This algorithm works on a specific path, starting from the start location and ending in the location to check. However, with the presence of loops, the number of paths is infinite. A location deemed as unreachable can therefore still be reachable, following a different path. Still, this factor can be included in the coverage statistics: the number of reachability checks to each unreachable location can be reported. This allows the tester to determine whether additional testing to these 'unreachable' locations is necessary.
