\section{Graph Grammars}\label{sec:graph}
A \textit{Graph Grammar} (GG) is also a specification of system behavior like LTSs and STSs are. A GG is composed of a set of graph transformation rules. These rules indicate how a graph can be transformed to a new graph. These graphs are called \textit{host graphs}. The rules are composed of graphs themselves, which are called \textit{rule graphs}.

The rest of this section is ordered as follows: first, graphs, host graphs, rule graphs and graph transformation rules are explained. Then, the definition of a \textit{Graph Transition System} (GTS) is given. An example of a GG and a GTS is then given. Finally, the definition of IOGGs is given. For a more detailed overview of GGs, we refer to~\cite{Rensink:graph_grammars, Heckel2006187, Andries1999}.

\vspace{10px}
\begin{definition} Graphs \\
A \textit{graph} is composed of nodes and edges. In this report, we assume a universe of nodes $\Nodes = \GraphNodes \uplus \mathbb{U} \uplus \Variables \uplus 2^\Terms$\newnot{symbol:Nodes}, where $\GraphNodes$\newnot{symbol:GraphNodes} is the universe of standard graph nodes. $\Edges$\newnot{symbol:Edges} is the universe of edges between two nodes in $\Nodes$.
\end{definition}

\vspace{10px}
\begin{definition} Host graphs \\
A host graph $\hostGraph$ is a tuple $\langle \DefinedHostNodes, \DefinedHostEdges\rangle$\newnot{symbol:hostGraph}, where:
\begin{itemize}
  \item $\DefinedHostNodes \subseteq (\mathbb{W} \uplus \mathbb{U})$ is the node set of $\hostGraph$
  \item $\DefinedHostEdges \subseteq (\DefinedHostNodes\backslash\mathbb{U} \times \Labels \times \DefinedHostNodes)$ is the edge set of $\hostGraph$
\end{itemize}
\end{definition}
\vspace{10px}

Figure~\ref{fig:hostgraph_example} shows an example of a host graph. Here, $n_1, n_2 \in \mathbb{W}$ are the \textit{identities} of the nodes. The other four nodes are values in $\mathbb{U}$.

\begin{figure}[ht]
  \begin{center}
    $\xymatrix{
   \fbox{1} & \fbox{$n_1$} \ar[l]^{b} \ar[d]^{d} \ar[r]^{a} & \fbox{$n_2$} \ar[r]^{c} \ar[d]^{e} & \fbox{false} \\
   & \fbox{4.2} & \fbox{"string"}
}$

  \end{center}
  \caption{An example of a host graph}
  \label{fig:hostgraph_example}
\end{figure}

\vspace{10px}
\begin{definition} Rule graphs \\
A rule graph $\ruleGraph$ is a tuple $\langle \DefinedRuleNodes, \DefinedRuleEdges \rangle$\newnot{symbol:ruleGraph}, where:
\begin{itemize}
  \item $\DefinedRuleNodes \subseteq (\Nodes \backslash \mathbb{U})$ is the node set of $\ruleGraph$
  \item $\DefinedRuleEdges \subseteq (\DefinedRuleNodes \times \Labels \times \DefinedRuleNodes)$ is the edge set of $\ruleGraph$
\end{itemize}
In addition, the following must hold:
\begin{itemize}
\item $\forall \node \in \DefinedRuleNodes \cap 2^\Terms\: .\:\mathit{var}(\node) \subseteq \DefinedRuleNodes$ - The variables used in the terms must be present as nodes in the rule graph.
\item $\forall \node \in \DefinedRuleNodes \cap \node \in \Variables\: .\: \exists (\_, \_, \node) \in \DefinedRuleEdges$ - If a variable is used in a rule graph, it needs context. Therefore, there must be an edge with the variable node as target.
\end{itemize}
\end{definition}
\vspace{10px}

Figure~\ref{fig:rulegraph_example} shows an example of a rule graph. Here, $r_1, r_2 \in \mathbb{W}$ are the node identities, $x_1,x_2 \in \Variables^{int}$ and $\{x_1+1, x_2\} \in 2^\Terms$. The set of terms is mapped as a node to the same value. This mapping is explained in the next definition. The consequence is that this node implicitly expresses the relation $x_1+1 = x_2$.

\begin{figure}[ht]
  \begin{center}
    $\xymatrix{
   \fbox{$int:x_1$} & \fbox{$r_1$} \ar[l]^{b} \ar[d]^{d} \ar[r]^{a} & \fbox{$r_2$} & \fbox{$x_1+1, x_2$} \\
   & \fbox{$int:x_2$}
}$

  \end{center}
  \caption{An example of a rule graph}
  \label{fig:rulegraph_example}
\end{figure}

\vspace{10px}
\begin{definition} Morphisms \\
A graph $g$ has a \textit{morphism} to a graph $g'$ if there is a structure-preserving mapping from the nodes and the edges of $g$ to the nodes and the edges of $g'$ respectively. A graph $g$ has a partial morphism to a graph $g'$ if there are elements in $g$ without an image in $g'$.
\end{definition}
\vspace{10px}

\begin{definition} Images \\
Given a morphism from a graph $g$ to a graph $g'$, a node or edge $\node$ in $g$ has an \textit{image} $\node'$ in $g'$. $\node$ is then a \textit{pre-image} of $\node'$. For each type of node, an explanation is given:
\begin{itemize}
  \item A variable $\variable \in \Variables^{\sort}, \sort \in \Sorts$, has an image $i$ in a host graph if $i \in \mathbb{U}^{\sort}$. 
  \item A node $\node \in 2^\Terms$ has an image $i$ in a host graph if $i$ is the valuation of all terms in $\node$. 
\end{itemize}
\end{definition}

\vspace{10px}
\begin{definition} Transformation rules \\
A transformation rule $\ggrule$\newnot{symbol:ggrule} is a tuple $\langle \mathit{LHS}, \mathit{NAC}, \mathit{RHS}, \ltsLabel \rangle$, where:
\begin{itemize}
  \item $\mathit{LHS}$ is a rule graph representing the left-hand side of the rule
  \item $\mathit{NAC}$ is a set of rule graphs representing the negative application conditions
  \item $\mathit{RHS}$ is a rule graph representing the right-hand side of the rule
  \item $\ltsLabel \in \Labels$ is the label of the rule
\end{itemize}
There exist implicit partial morphisms from the $\mathit{LHS}$ to each rule graph in $\mathit{NAC}$ and from the $\mathit{LHS}$ to the $\mathit{RHS}$ by means of the node identities. These morphisms are \textit{rule graph morphisms}.
\end{definition}
\vspace{10px}

\begin{definition} Creators and erasers \\
A \textit{creator} is an edge or node in the $\mathit{RHS}$ of a rule, that is not in the $\mathit{LHS}$ of the rule. An \textit{eraser} is an edge or node in the $\mathit{LHS}$ of a rule that is not in the $\mathit{RHS}$ or a rule.
\end{definition}

\vspace{10px}
\begin{definition} Rule matches \\
A rule $\ggrule$ has a \textit{rule match} on a host graph $\hostGraph$ if its $\mathit{LHS}$ has a morphism to $\hostGraph$ and none of the graphs in $\mathit{NAC}$ have a morphism to $\hostGraph$. The morphism of the $\mathit{LHS}$ to a host graph is a \textit{match morphism}.
\end{definition}

\vspace{10px}
\begin{definition} Rule transitions \\
After the rule match is found, all nodes and edges in $\mathit{LHS}$ that do not have an image in $\mathit{RHS}$, are removed from $\hostGraph$. All elements in $\mathit{RHS}$ that do not have a pre-image in $\mathit{LHS}$, are added to $\hostGraph$. The result of this process is called a \textit{rule transition}, denoted by: $\hostGraph \xrightarrow{\ggrule, \rulematch}\hostGraph'$, where $\rulematch \in \RuleMatches$\newnot{symbol:RuleMatches} is the morphism of the $\mathit{LHS}$ to $\hostGraph$.
\end{definition}
\vspace{10px}

Figure \ref{fig:gg_example} shows an example of the initial graph $\startGraph$, one rule of a GG and the corresponding rule match. $\startGraph$ can be represented by $\langle\{n1,n2\},\{\langle n1,a,n1\rangle, \langle n1,A,n2\rangle,\langle n2,B,n2\rangle\}\rangle$. The $\mathit{LHS}$ of the rule has a match in $\startGraph$. Neither $\mathit{NAC1}$ and $\mathit{NAC2}$ have a match in $\startGraph$, because the edge with label $C$ does not exist in $\startGraph$. The new graph after applying the rule is $\hostGraph_1$.

\begin{figure}[ht]
  \begin{center}
    % To use this figure in your LaTeX document
% import the package groove/resources/groove2tikz.sty
%
% Special colors
\begin{tikzpicture}[
% Special color styles
scale=\tikzscale]
\node[node] (lhs)  at (2.200, 0.000) {\ml{\textbf{LHS} \\ $\xymatrix{
   r1 \ar@(dl,dr)[]_{A} \ar[r]^{a} & r2 \ar@(dl,dr)[]_{B}
}$
}};
\node[node] (n1)  at (-0.500, 0.000) {\ml{\textbf{NAC1} \\ $\xymatrix{
   \fbox{A} \ar[r]^{a} & \fbox{C}
}$
}};
\node[node] (n2)  at (-0.500, -1.500) {\ml{\textbf{NAC2} \\ $\xymatrix{
   \bullet \ar@(dl,dr)[]_{A} \ar[r]^{b} & \bullet \ar@(dl,dr)[]_{C}
}$
}};
\node[node] (g)  at (2.200, -2.000) {\ml{\textbf{$\startGraph$} \\ $\xymatrix{
   \fbox{A} \ar[r]^{a} & \fbox{B}
}$
}};
\node[node] (rhs)  at (5.000, 0.000) {\ml{\textbf{RHS} \\ $\xymatrix{
   \bullet \ar@(dl,dr)[]_{A} \ar[r]^{b} & \bullet \ar@(dl,dr)[]_{B}
}$
}};
\node[node] (g2)  at (5.000, -2.000) {\ml{\textbf{$\hostGraph_1$} \\ $\xymatrix{
   n1 \ar@(dl,dr)[]_{A} \ar[r]^{b} & n2 \ar@(dl,dr)[]_{B}
}$
}};
\path[edge] (lhs)  -- node[lab]{partial morphism} (n1) ;
\path[edge] (lhs)  -- node[lab]{partial morphism} (rhs) ;
\path[edge] (lhs)  -- node[lab]{partial morphism} (n2) ;
\path[edge] (g)  -- node[lab]{trafo} (g2) ;
\path[edge](lhs.south -| 2.200, -2.000) -- node[lab]{match} (g) ;
\path[edge](rhs.south -| 5.000, -2.000) -- node[lab]{match} (g2) ;
\userdefinedmacro
\end{tikzpicture}
\renewcommand{\userdefinedmacro}{\relax}

  \end{center}
  \caption{An example of a graph transformation}
  \label{fig:gg_example}
\end{figure}

\vspace{10px}
\begin{definition} Graph Grammars \\
A graph grammar is a tuple $\langle \Rules, \startGraph \rangle$, where:
\begin{itemize}
\item $\Rules$\newnot{symbol:Rules} is a set of graph transformation rules
\item $\startGraph$\newnot{symbol:startGraph} is the initial graph
\end{itemize}
\end{definition}
\vspace{10px}

By repeatedly applying graph transformation rules to the start graph and all its consecutive graphs, a GG can be explored to reveal a \textit{Graph Transition System} (GTS). This transition system consists of graphs connected by rule transitions.

\vspace{10px}
\begin{definition} Graph Transition Systems \\
A graph transition system is a tuple	$\langle \Graphs, \Rules, \RuleMatches, \RuleTransitions, \startGraph\rangle$, where:
\begin{itemize}
\item $\Graphs$\newnot{symbol:Graphs} is a set of graphs
\item $\Labels \subseteq \Rules \times \RuleMatches$ is a set of labels
\item $\RuleTransitions \subseteq \Graphs \times \Labels \times \Graphs$\newnot{symbol:RuleTransitions} is the rule transition relation
\item $\startGraph \in \Graphs$ is the initial graph
\end{itemize}
\end{definition}
\vspace{10px}

Let $K = \langle \Rules, \startGraph \rangle$. A GTS $O = \langle \Graphs, \Rules, \RuleMatches, \RuleTransitions, \startGraph\rangle$ is derived from $K$ by the following. $\Graphs, \RuleMatches, \RuleTransitions$ are the smallest sets, such that:
\begin{itemize}
\item $\startGraph \in \Graphs$
\item if $\hostGraph \in \Graphs$ and $\hostGraph \xrightarrow{\ggrule, \rulematch}\hostGraph'$ then $\hostGraph' \in \Graphs, (\ggrule, \rulematch) \in \Labels, (\hostGraph \xrightarrow{\ggrule, \rulematch}\hostGraph') \in \RuleTransitions$
\end{itemize}

In order to specify stimuli and responses with GGs, a definition is given for an \textit{Input-Output GG} (IOGG).
\vspace{10px}
\begin{definition} Input-Output Graph Grammars \\
An IOGG is a tuple $\langle\IORules, \startGraph\rangle$\newnot{symbol:IORules}, where $\IORules \subseteq \Rules \times \IOTypes$ are input-output transformation rules.
\end{definition}
\vspace{10px}

Exploring an IOGG leads to an \textit{Input-Output Graph Transition System} (IOGTS).
\vspace{10px}
\begin{definition} Input-Output Graph Transition Systems \\
An IOGTS is a tuple  $\langle \Graphs, \IORules, \RuleMatches, \IORuleTransitions, \startGraph\rangle$, where:
\begin{itemize}
\item \newnot{symbol:IOLabels}$\IOLabels \subseteq \IORules \times \RuleMatches$ are the input-output labels
\item \newnot{symbol:IORuleTransitions}$\IORuleTransitions \subseteq \Graphs \times \IOLabels \times \Graphs$ are the input-output rule transitions
\end{itemize}
\end{definition}

\vspace{10px}
\begin{definition} Rule priorities \\
A graph grammar with \textit{rule priorities} $\RulePriorities$ assigns a priority $\rulePriority\in\mathbb{N}$ to a $\ggrule \in \Rules$, such that ${\ggrule \mapsto \rulePriority}\in\RulePriorities$. The definition of GTSs is extended with this definition of rule priorities and the following:\\
\vspace{5px}
$\ggrule_1, \ggrule_2\in \Rules, \hostGraph \in \Graphs . (\hostGraph \xrightarrow{\ggrule_1, \rulematch_1}\hostGraph') \in \RuleTransitions \land \RulePriorities(\ggrule_1) > \RulePriorities(\ggrule_2) \rightarrow (\hostGraph \xrightarrow{\ggrule_2, \rulematch_2}\hostGraph') \notin \RuleTransitions$\\
\end{definition}
\vspace{10px}
In the graph grammar in Figure~\ref{fig:priority_gg}, the `add' rule produces a rule transition to a graph, where the $sub$ rule produces a rule transition back to the start graph. Suppose $\RulePriorities(add) > \RulePriorities(sub)$, then the `sub' rule does not have a outgoing rule transition from the start graph.

\begin{figure}[ht]
  \begin{center}
    \subfloat[The initial graph]{\label{fig:priority_start}$\xymatrix{
  \fbox{A} \ar[r]^{a} & \fbox{1}
}$}\\
    \subfloat[The $LHS$ of the $add$ rule]{\label{fig:priority_add_lhs}$\xymatrix{
  \fbox{A}\ar[r]^{a} & \fbox{int: x} & \fbox{x < 2}
}$}\hspace{20px}
    \subfloat[The $RHS$ of the $add$ rule]{\label{fig:priority_add_rhs}$\xymatrix{
  \fbox{A}\ar[r]^{a} & \fbox{x + 1}
}$}\\
    \subfloat[The $LHS$ of the $sub$ rule]{\label{fig:priority_sub_lhs}$\xymatrix{
  \fbox{A}\ar[r]^{a} & \fbox{int: x} & \fbox{x > 0}
}$}\hspace{20px}
    \subfloat[The $RHS$ of the $sub$ rule]{\label{fig:priority_sub_rhs}$\xymatrix{
  \fbox{A}\ar[r]^{a} & \fbox{x - 1}
}$}
  \end{center}
  \caption{Priority rules}
  \label{fig:priority_gg}
\end{figure}