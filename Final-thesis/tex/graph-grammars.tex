\section{Graph Grammars}\label{sec:graph}
A \textit{Graph Grammar} (GG) is composed of a set of graph transformation rules. These rules indicate how a graph can be transformed to a new graph. These graphs are called \textit{host graphs}. The rules are composed of graphs themselves, which are called \textit{rule graphs}.

The rest of this section is ordered as follows: first, host graphs, rule graphs and graph transformation rules are explained. Then, the definition of a \textit{Graph Transition System} (GTS) is given. An example of a GG and a GTS is then given. Finally, the definition of IOGGs is given. For a more detailed overview of GGs, we refer to~\cite{Rensink:graph_grammars, Heckel2006187, Andries1999}.

\subsection{Host graphs}
In this report, we assume a universe of nodes $\Nodes = \mathbb{W} \uplus \mathbb{U} \uplus \Variables \uplus 2^\Terms$\newnot{symbol:Nodes}.

A host graph $\hostGraph$ is a tuple $\langle \DefinedNodes, \DefinedEdges\rangle$\newnot{symbol:hostGraph}, where:
\begin{itemize}
  \item $\DefinedNodes \subseteq (\mathbb{W} \uplus \mathbb{U})$
  \item $\DefinedEdges$ is a set of edges, where each $\edge \in \DefinedEdges$ has a label and source node $\node_s \in \DefinedNodes\backslash\mathbb{U}$ and target node $\node_t\in\DefinedNodes$
\end{itemize}

\subsection{Rule graphs}
A rule graph $\ruleGraph$ is a tuple $\langle \DefinedRuleNodes, \DefinedRuleEdges \rangle$\newnot{symbol:ruleGraph}, where:
\begin{itemize}
  \item $\DefinedRuleNodes \subseteq (\Nodes \backslash \mathbb{U})$
  \item $\DefinedRuleEdges$ is a set of edges, where each $\edge \in \DefinedRuleEdges$ has a label and source and target nodes in $\DefinedRuleNodes$
\end{itemize}

\subsection{Morphisms \& matches}
A graph $g$ has a \textit{morphism} to a graph $g'$ if there is a structure-preserving mapping from the nodes and the edges of $g$ to the nodes and the edges of $g'$ respectively. Such a mapping is called a \textit{morphism}. A node or edge $\node$ in graph $g$ is then said to have an \textit{image} in graph $g'$ and $\node$ is a \textit{pre-image} of the image. A \textit{match} is defined by a morphism from a rule graph to a host graph. A variable $\variable \in \Variables^{\sort}, \sort \in \Sorts$, has an image $i$ in a host graph if $i \in \mathbb{U}^{\sort}$. A node $\node \in 2^\Terms$ has an image $i$ in a host graph if $i$ is the valuation of all terms in $\Terms$. A graph $g$ has a partial morphism to a graph $g'$ if there are elements in $g$ without an image in $g'$.

\subsection{Graph transformation rules}\label{sec:graph_rules}
\vspace{5px}
\begin{definition}
A transformation rule is a tuple $\langle \mathit{LHS}, \mathit{NAC}, \mathit{RHS}\rangle$, where:
\begin{itemize}
  \item $\mathit{LHS}$ is a rule graph representing the left-hand side of the rule
  \item $\mathit{NAC}$ is a set of rule graphs representing the negative application conditions
  \item $\mathit{RHS}$ is a rule graph representing the right-hand side of the rule
\end{itemize}
\end{definition}

\begin{comment}\begin{definition}
$\mathit{LHS}$ is a tuple $\langle \mathcal{V}_{LHS}, \mathcal{E}_{LHS}, \mathcal{O}_{LHS} \rangle$, where $\mathcal{O}_{LHS} \subseteq \mathcal{V}^n \times O_p \times \mathcal{V}$.
\end{definition}\end{comment}

A rule $R$ is applicable on a host graph $\hostGraph$ if its $\mathit{LHS}$ has a match in $\hostGraph$ and $\not\exists\,n \in \mathit{NAC}$ such that $n$ has a match in $\hostGraph$ and $\forall\,e \in \mathit{LHS}$, if $e$ has an image $i$ in $n$, and an image $j$ in $\hostGraph$, then $j$ should be an image of $i$.

This 'applicability' as defined here is refered to as a \textit{rule match}. After the rule match is applied to the graph, all elements in $\mathit{LHS}$ that do not have an image in $\mathit{RHS}$, are removed from $\hostGraph$ and all elements in $\mathit{RHS}$ that do not have a pre-image in $\mathit{LHS}$, are added to $\hostGraph$. This process, called a \textit{rule transition}, is denoted $\hostGraph \xrightarrow{\ggrule, \rulematch}\hostGraph'$, where $\ggrule$ is the rule and $\rulematch$\newnot{symbol:RuleMatches} is the rule match, i.e. the morphism of the $\mathit{LHS}$ to $\hostGraph$.

\subsection{Graph Transition Systems}
By repeatedly applying graph transformation rules to the start graph and all its consecutive graphs, a GG can be explored to reveal a \textit{Graph Transition System} (GTS). This transition system consists of graphs connected by rule transitions.
\vspace{5px}
\begin{definition}
A graph transition system is a tuple	$\langle \Graphs, \Rules, \RuleMatches, \RuleTransitions, \startGraph\rangle$, where:
\begin{itemize}
\item $\Graphs$\newnot{symbol:Graphs} is a set of graphs
\item $\Rules$\newnot{symbol:Rules} is a set of transformation rules
\item $\RuleMatches$ is a set of rule matches
\item $\RuleTransitions \in \Graphs \times \Rules \times \RuleMatches \times \Graphs$\newnot{symbol:RuleTransitions} is the rule transition relation
\item $\startGraph \in \Graphs$\newnot{symbol:startGraph} is the initial graph
\end{itemize}
\end{definition}

\subsection{Example}\label{sec:gts_example}
Figure \ref{fig:gg_example} shows an example of the initial graph $\startGraph$ and one rule of a GG. $\startGraph$ can be represented by $\langle\{n1,n2\},\{\langle n1,a,n1\rangle, \langle n1,A,n2\rangle,\langle n2,B,n2\rangle\}\rangle$. The $\mathit{LHS}$ of the rule has a match in $\startGraph$. Neither $\mathit{NAC1}$ and $\mathit{NAC2}$ have an match in $\startGraph$, because the edge with label $C$ does not exist in $\startGraph$. The new graph after applying the rule is $\hostGraph_1$.

\begin{figure}[ht]
  \begin{center}
    % To use this figure in your LaTeX document
% import the package groove/resources/groove2tikz.sty
%
% Special colors
\begin{tikzpicture}[
% Special color styles
scale=\tikzscale]
\node[node] (lhs)  at (2.200, 0.000) {\ml{\textbf{LHS} \\ $\xymatrix{
   r1 \ar@(dl,dr)[]_{A} \ar[r]^{a} & r2 \ar@(dl,dr)[]_{B}
}$
}};
\node[node] (n1)  at (-0.500, 0.000) {\ml{\textbf{NAC1} \\ $\xymatrix{
   \fbox{A} \ar[r]^{a} & \fbox{C}
}$
}};
\node[node] (n2)  at (-0.500, -1.500) {\ml{\textbf{NAC2} \\ $\xymatrix{
   \bullet \ar@(dl,dr)[]_{A} \ar[r]^{b} & \bullet \ar@(dl,dr)[]_{C}
}$
}};
\node[node] (g)  at (2.200, -2.000) {\ml{\textbf{$\startGraph$} \\ $\xymatrix{
   \fbox{A} \ar[r]^{a} & \fbox{B}
}$
}};
\node[node] (rhs)  at (5.000, 0.000) {\ml{\textbf{RHS} \\ $\xymatrix{
   \bullet \ar@(dl,dr)[]_{A} \ar[r]^{b} & \bullet \ar@(dl,dr)[]_{B}
}$
}};
\node[node] (g2)  at (5.000, -2.000) {\ml{\textbf{$\hostGraph_1$} \\ $\xymatrix{
   n1 \ar@(dl,dr)[]_{A} \ar[r]^{b} & n2 \ar@(dl,dr)[]_{B}
}$
}};
\path[edge] (lhs)  -- node[lab]{partial morphism} (n1) ;
\path[edge] (lhs)  -- node[lab]{partial morphism} (rhs) ;
\path[edge] (lhs)  -- node[lab]{partial morphism} (n2) ;
\path[edge] (g)  -- node[lab]{trafo} (g2) ;
\path[edge](lhs.south -| 2.200, -2.000) -- node[lab]{match} (g) ;
\path[edge](rhs.south -| 5.000, -2.000) -- node[lab]{match} (g2) ;
\userdefinedmacro
\end{tikzpicture}
\renewcommand{\userdefinedmacro}{\relax}

  \end{center}
  \caption{An example of a GG}
  \label{fig:gg_example}
\end{figure}

\begin{comment}
\begin{figure}[ht]
  \begin{center}
    \subfloat[The initial graph]{\label{fig:gg_graph}$\xymatrix{
   \fbox{A} \ar[r]^{a} & \fbox{B}
}$
}\hspace{20px}
    \subfloat[The LHS]{\label{fig:gg_lhs}$\xymatrix{
   r1 \ar@(dl,dr)[]_{A} \ar[r]^{a} & r2 \ar@(dl,dr)[]_{B}
}$
}\hspace{20px}
    \subfloat[The first NAC]{\label{fig:gg_nac1}$\xymatrix{
   \fbox{A} \ar[r]^{a} & \fbox{C}
}$
}
    
    \subfloat[The second NAC]{\label{fig:gg_nac2}$\xymatrix{
   \bullet \ar@(dl,dr)[]_{A} \ar[r]^{b} & \bullet \ar@(dl,dr)[]_{C}
}$
}\hspace{20px}
    \subfloat[The RHS]{\label{fig:gg_rhs}$\xymatrix{
   \bullet \ar@(dl,dr)[]_{A} \ar[r]^{b} & \bullet \ar@(dl,dr)[]_{B}
}$
}\hspace{20px}
    \subfloat[The result]{\label{fig:gg_result}$\xymatrix{
   n1 \ar@(dl,dr)[]_{A} \ar[r]^{b} & n2 \ar@(dl,dr)[]_{B}
}$
}
  \end{center}
  \caption{An example of a GG}
  \label{fig:gts}
\end{figure}\marginpar{This picture needs to be placed in a structure, need a way of nesting graphs...}
\end{comment}

\subsection{Input-Output GGs}
In order to specify stimuli and responses with GGs, a definition is given for an \textit{Input-Output GG} (IOGG). Concretely, the IOGG places input and output labels on its rule transitions. Following the definition from IOLTSs, each rule transition label $l \in L$ has a type $\iotype \in \IOTypes$. Exploring an IOGG leads to an \textit{Input-Output Graph Transition System} (IOGTS).
