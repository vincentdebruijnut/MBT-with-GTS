\section{Graph Grammars}\label{sec:graph}
A \textit{Graph Grammar} (GG) is composed of a set of graph transformation rules. These rules indicate how a graph can be transformed to a new graph. These graphs are called \textit{host graphs}. The rules are composed of graphs themselves, which are called \textit{rule graphs}.

The rest of this section is ordered as follows: first, host graphs are explained. This is then used to explain graph transformation rules. Then, the definition of a \textit{Graph Transition System} (GTS) is given. An example of a GG and a GTS is then given. Finally, a method for transforming a GTS to an STS is given. For a more detailed overview of GGs, we refer to~\cite{Rensink:graph_grammars, Heckel2006187, Andries1999}.

\subsection{Graphs and morphisms}
A graph $\mathcal{G}^{graph}$ is a tuple $\langle V, E\rangle$, where:
\begin{itemize}
  \item $V$ is a set of nodes
  \item $E$ is a set of edges, where each $e \in E$ has a label and source and target nodes in $N$
\end{itemize}

A graph $H$ has an \textit{occurrence} in a graph $G$, denoted by $H \rightarrow G$, if there is a mapping from the nodes and the edges of $H$ to the nodes and the edges of $G$ respectively. Such a mapping is called a \textit{morphism}. A node or edge $x$ in graph $H$ is then said to have an \textit{image} in graph $G$ and $x$ is a \textit{pre-image} of the image. A graph $H$ has a partial morphism to a graph $G$ if there are elements in $H$ without an image in $G$.

\subsection{Host graphs}
A host graph, given an algebra $\mathcal{A}$, is given by: $\mathcal{G}_\mathcal{A}^{host} \subseteq \mathcal{G}^{graph} \uplus \mathbb{U}_\mathcal{A}^{int} \uplus \mathbb{U}_\mathcal{A}^{real} \uplus \mathbb{U}_\mathcal{A}^{bool} \uplus \mathbb{U}_\mathcal{A}^{string}$.

\subsection{Graph transformation rules}\label{sec:graph_rules}
A rule graph is given by $\mathcal{G}^{rule} \subseteq \mathcal{G}^{graph} \uplus F_{constant}$. A variable $v \in \mathcal{V}^{s}, s \in S$ can have an image $i$ in a host graph if $i \in \mathbb{U}_\mathcal{A}^{s}$. A constant symbol $c \notin \mathcal{V}$ can have an image $i$ in a host graph if $i \in \mathbb{U}_\mathcal{A}^{s}$ and $i$ is represented by the constant symbol $c$ in the signature.
\vspace{5px}
\begin{definition}
A transformation rule is a tuple $\langle \mathit{LHS}, \mathit{NAC}, \mathit{RHS}\rangle$, where:
\begin{itemize}
  \item $\mathit{LHS}$ is a rule graph representing the left-hand side of the rule
  \item $\mathit{NAC}$ is a set of rule graphs representing the negative application conditions
  \item $\mathit{RHS}$ is a rule graph representing the right-hand side of the rule
\end{itemize}
\end{definition}

\begin{comment}\begin{definition}
$\mathit{LHS}$ is a tuple $\langle \mathcal{V}_{LHS}, \mathcal{E}_{LHS}, \mathcal{O}_{LHS} \rangle$, where $\mathcal{O}_{LHS} \subseteq \mathcal{V}^n \times O_p \times \mathcal{V}$.
\end{definition}\end{comment}

A rule $R$ is applicable on a graph $G$ if its $\mathit{LHS}$ has an occurrence in $G$ and $\not\exists\,n \in \mathit{NAC}$ such that $n$ has an occurence in $G$ and $\forall\,e \in \mathit{LHS}$, if $e$ has an image $i$ in $n$, and an image $j$ in $G$, then $j$ should be an image of $i$.

This 'applicability' as defined here is refered to as a rule \textit{match}. After the rule match is applied to the graph, all elements in $\mathit{LHS}$ that do not have an image in $\mathit{RHS}$, are removed from $G$ and all elements in $\mathit{RHS}$ that do not have a pre-image in $\mathit{LHS}$, are added to $G$.

\subsection{Graph Transition Systems}
By repeatedly applying graph transformation rules to the start graph and all its consecutive graphs, a GG can be explored to reveal a \textit{Graph Transition System} (GTS). This transition system consists of \textit{graph states} connected by \textit{rule transitions}.
\vspace{5px}
\begin{definition}
A graph transition system is an 8-tuple	$\langle G, R, L, T_{gts}, g_0\rangle$, where:
\begin{itemize}
\item $G$ is a set of host graphs
\item $R$ is a set of transformation rules
\item $L$ is a finite set of labels\marginpar{A: what are the labels? V: hoe bedoel je? strings?}
\item $T_{gts} \in G \times (L \cup \{\tau\}) \times G \times R$, with $\tau \notin L$, is the rule transition relation
\item $g_0 \in G$ is the initial graph
\end{itemize}
We write $s \xrightarrow{\mu}s'$ if there is a rule transition labelled $\mu$ from state s to state s', i.e., $(s, \mu, s') \in T$.
\end{definition}

These systems are very similar to LTSs. A GTS can be transformed to an LTS by omitting the graphs, rules and mappings.

\subsection{Example}\label{sec:gts_example}
Figure \ref{fig:gts} shows an example of the initial graph and one rule of a GG. The initial graph can be represented by $\langle\{n1,n2\},\{\langle n1,a,n1\rangle, \langle n1,A,n2\rangle,\langle n2,B,n2\rangle\}\rangle$. The $\mathit{LHS}$ of the rule has an occurrence in the initial graph. None of the graphs in the $\mathit{NAC}$ have an occurrence in the initial graph, because the edge with label $C$ does not exist in the initial graph. The new graph after applying the rule is in Figure~\ref{fig:gg_result}.

\begin{figure}[ht]
  \begin{center}
    \subfloat[The initial graph]{\label{fig:gg_graph}$\xymatrix{
   n1 \ar@(dl,dr)[]_{A} \ar[r]^{a} & n2 \ar@(dl,dr)[]_{B}
}$
}\hspace{20px}
    \subfloat[The LHS]{\label{fig:gg_lhs}$\xymatrix{
   r1 \ar@(dl,dr)[]_{A} \ar[r]^{a} & r2 \ar@(dl,dr)[]_{B}
}$
}\hspace{20px}
    \subfloat[The first NAC]{\label{fig:gg_nac1}$\xymatrix{
   \bullet \ar@(dl,dr)[]_{A} \ar[r]^{a} & \bullet \ar@(dl,dr)[]_{C}
}$
}
    
    \subfloat[The second NAC]{\label{fig:gg_nac2}$\xymatrix{
   \bullet \ar@(dl,dr)[]_{A} \ar[r]^{b} & \bullet \ar@(dl,dr)[]_{C}
}$
}\hspace{20px}
    \subfloat[The RHS]{\label{fig:gg_rhs}$\xymatrix{
   \fbox{A} \ar[r]^{b} & \fbox{B}
}$
}\hspace{20px}
    \subfloat[The result]{\label{fig:gg_result}$\xymatrix{
   n1 \ar@(dl,dr)[]_{A} \ar[r]^{b} & n2 \ar@(dl,dr)[]_{B}
}$
}
  \end{center}
  \caption{An example of a GG}
  \label{fig:gts}
\end{figure}\marginpar{This picture needs to be placed in a structure, need a way of nesting graphs...}

\subsection{Input-Output GGs}
In order to specify stimuli and responses with GGs, a definition is given for an \textit{Input-Output GG} (IOGG). Concretely, the IOGG places input and output labels on its rule transitions. Following the definition from IOLTSs, each rule transition label $l \in L$ has a type $t \in T$, where $T = \{input, output\}$. Exploring an IOGG leads to an \textit{Input-Output Graph Transition System} (IOGTS).
