\section{Algebra}\label{sec:algebra}

Some basic concepts from algebra are described here. For a general introduction into logic we refer to~\cite{Huth:logic}. This section explains in order: multi-sorted signatures, algebrae, variables \& terms and term-mapping \& valuations. The algebra described here will be used in the next sections to formally define Symbolic Transition Systems and Graph Grammars.

\vspace{10px}\begin{definition} Multi-sorted Signatures \\
A \textit{multi-sorted signature} $\langle \Sorts, \FunctionSymbols \rangle$ describes the sorts and function symbols of a formal language. $\Sorts$\newnot{symbol:Sorts} is a set of sorts. $\FunctionSymbols$\newnot{symbol:FunctionSymbols} is a set of function symbols. Each $\functionSymbol \in \FunctionSymbols$ has an arity $n \in \mathbb{N}$, where a function symbol with arity $n = 0$ is called a constant symbol. $\FunctionSymbols^i$ denotes the subset of $\FunctionSymbols$ with function symbols of arity $n = i$. The sort of a function symbol $\functionSymbol \in \FunctionSymbols$ with arity $n$ is given by $\sigma(f) = \sort_1 ... \sort_{n+1}$, with $\sort_i \in \Sorts$ for $1 \leq i \leq n$. $\sort_{n+1}$ is the return sort. In this report, $\Sorts =  \{\mathit{int, real, bool, string}\}$ denoting the integer, real, boolean and string sorts respectively. $\FunctionSymbols$ features the commonly used function symbols, which include, but are not restricted to, `int:+', `string:=', `$\neg$', `1', which are the addition of integers, the equality of strings, the negation of a boolean and the integer `one' respectively. The sorts and arities of these examples are given by:
\begin{enumerate}
\item $\sigma(\mathit{int:}+) = \langle int, int, int\rangle$
\item $\sigma(\mathit{string:}=) = \langle string, string, bool\rangle$
\item $\sigma(\neg) = \langle bool, bool\rangle$
\item $\sigma(1) = \langle int \rangle$
\end{enumerate}
\end{definition}

\vspace{10px}
\begin{definition} Algebrae \\
An \textit{algebra} $\Algebra = \langle \mathbb{U}, \Functions\rangle$\newnot{symbol:Algebra} has a non-empty set $\mathbb{U}$\newnot{symbol:Universe} of values called a \textit{universe}, partitioned into $\mathbb{U}^\sort$ for each $\sort \in \Sorts$, and a set $\Functions$\newnot{symbol:Functions} of functions. A function $\function_\Algebra$ is typed $\mathbb{U}_\Algebra^{\sort_1} \times ... \mathbb{U}_\Algebra^{\sort_n} \rightarrow \mathbb{U}_\Algebra^{\sort_{n+1}}$, where $\sort_1 ... \sort_{n+1}$ is the sort of the function symbol given by the signature. For example, $<_\Algebra: \mathbb{U}_\Algebra^{int} \times \mathbb{U}_\Algebra^{int} \rightarrow \mathbb{U}_\Algebra^{bool}$ represents the `less-than' comparison of two integers.
\end{definition}

\vspace{10px}
\begin{definition} Point algebra \\
We define a \textit{point algebra} $\PointAlgebra$\newnot{symbol:PointAlgebra} to be an algebra with $\forall \sort \in \Sorts . |\mathbb{U}_\PointAlgebra^\sort| = 1$.
\end{definition}

\vspace{10px}
\begin{definition} Variables \\
We define $\Variables = \Variables^{int} \uplus \Variables^{real} \uplus \Variables^{bool} \uplus \Variables^{string}$\newnot{symbol:Variables} to be the set of \textit{variables}. \textit{Terms} over $\DefinedVariables$, denoted $\Terms(\DefinedVariables)$\newnot{symbol:Terms}, are built from function symbols $\FunctionSymbols$ and variables $\DefinedVariables \subseteq \Variables$. The definition of a term is:
\vspace{5px}\\
$\begin{array}{lrlr}\term & ::= & \functionSymbol(\term_1 ... \term_n) & ,\: where\: $n$\: is\: the\: arity\: of\: \function \\ 
& | & x&\mathit{,\: where\: x\: is\: a\: variable.}
\end{array}$
\vspace{5px}\\
We write $var(\term)$ to denote the set of variables appearing in a term $\term \in \Terms(\DefinedVariables)$. Terms $\term\in \Terms(\emptyset)$ are called ground terms. An example of a term $\term$ is $(x+(y-1))$, with $var(\term) = \{x,y\}$. The type of a term is given by:
\vspace{5px}\\
$\begin{array}{lll}\sortFunction: \term \mapsto & \sort       & \mathit{if}\: \term = x \in \Variables^\sort \\ 
 & \sort_{n+1} & \mathit{if}\: \term = \functionSymbol(\term_1 ... \term_n) \mathit{\:and\:} \sortFunction(\functionSymbol) = \sort_1 ... \sort_{n+1}\mathit{,\:provided\:} \sortFunction(\term_i) = \sort_i
\end{array}$
\vspace{5px}\\
The set of terms with return type $\mathit{bool}$, is denoted as $\BooleanTerms(\Variables)$\newnot{symbol:BooleanTerms}. An example is $(x < y)$, where the result is $\mathit{true}$ or $\mathit{false}$.
\end{definition}

\vspace{10px}
\begin{definition} Term-mapping \\
A \textit{term-mapping} is a function $\termMapping:\Variables \rightarrow \Terms(\Variables)$\newnot{symbol:termMapping}. A \textit{valuation} $\valuation$\newnot{symbol:valuation} is a function $\valuation:\Variables \rightarrow \mathbb{U}$ that assigns values to variables. For example, given an algebra, $\valuation:\{(x \mapsto 1), (y \mapsto 2)\}$ assigns the values 1 and 2 to the variables $x$ and $y$ respectively.
A valuation of a term given $\Algebra$ is defined by:
\vspace{5px}\\
$\begin{array}{llcl}
\valuation: & x & \mapsto & \valuation(x)\\
 & \functionSymbol(\term_1 ... \term_n) & \mapsto & \functionSymbol_\Algebra(\valuation(\term_1) ... \valuation(\term_n))
 \end{array}$
\vspace{5px}\\
When every variable in a term is defined by a valuation, the term can be valuated to a value. Therefore, when every variable in a term-mapping is defined by a valuation, a new valuation can be obtained. Formally, this is defined as: $\_{after}\_: (\Variables \rightarrow \mathbb{U}) \times (\Variables \rightarrow \Terms(\Variables)) \rightarrow (\Variables \rightarrow \mathbb{U})$. Given a valuation $\valuation$ and a term-mapping $\termMapping$, ($\valuation\:\mathit{after}\:\termMapping): \valuation \mapsto \valuation(\termMapping(\valuation))$.
\end{definition}