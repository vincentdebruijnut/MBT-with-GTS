\section{Algebra}\label{sec:algebra}

Some basic concepts from algebra are described here. For a general introduction into logic we refer to~\cite{Huth:logic}.

A \textit{multi-sorted signature} $\langle \Sorts, \FunctionSymbols \rangle$ describes the function symbols and sorts of a formal language. $\FunctionSymbols$\newnot{symbol:FunctionSymbols} is a set of function symbols. $\Sorts$\newnot{symbol:Sorts} is a set of sorts. Each $\functionSymbol \in \FunctionSymbols$ has an arity $n \in \mathbb{N}$, where a function symbol with arity $n = 0$ is called a constant symbol. $\FunctionSymbols^i$ denotes the subset of $\FunctionSymbols$, with function symbols of arity $n = i$. The sort of a function symbol $\functionSymbol \in \FunctionSymbols$ with arity $n$ is given by $\sigma(f) = \sort_1 ... \sort_n+1$, with $\sort_i \in \Sorts$ for $1 \leq i \leq n$. $\Sorts_{n+1}$ is the return sort. In this report, $\Sorts =  \{\mathit{int, real, bool, string}\}$ denoting the integer, real, boolean and string sorts respectively. $\FunctionSymbols$ features the commonly used function symbols, which include, but not restricted by, '+', '*', '=', '<', '0', '1'.

An \textit{algebra} $\Algebra = \langle \mathbb{U}, \Functions\rangle$ has a non-empty set $\mathbb{U}$ of constants called a \textit{universe}, partitioned into $\mathbb{U}^\sort$ for each $\sort \in \Sorts$, and a set $\Functions$\newnot{symbol:Functions} of functions. A function $\function_\Algebra$ is typed $\mathbb{U}_\Algebra^{\sort_1} \times ... \mathbb{U}_\Algebra^{\sort_n} \rightarrow \mathbb{U}_\Algebra^{\sort_{n+1}}$, where $\sort_1 ... \sort_{n+1}$ is the sort of the function symbol given by the signature. For example, $<_\Algebra: \mathbb{U}_\Algebra^{int} \times \mathbb{U}_\Algebra^{int} \rightarrow \mathbb{U}_\Algebra^{bool}$ represents the 'less-than' comparison of two integers.
 
We define $\Variables = \Variables^{int} \uplus \Variables^{real} \uplus \Variables^{bool} \uplus \Variables^{string}$\newnot{symbol:Variables} to be the set of \textit{variables}. \textit{Terms} over $\DefinedVariables$, denoted $\Terms(\DefinedVariables)$\newnot{symbol:Terms}, are built from function symbols $\FunctionSymbols$ and variables $\DefinedVariables \subseteq \Variables$. The definition of a term is:
\vspace{8px}\\
$\begin{array}{lrlr}\term & ::= & \function(\term_1 ... \term_n) &\\ & | & x&\mathit{,\: where\: x\: is\: a\: constant.}\end{array}$
\vspace{8px}\\
We write $var(\term)$ to denote the set of variables appearing in a term $\term \in \Terms(\DefinedVariables)$. Terms $\term\in \Terms(\emptyset)$ are called ground terms. An example of a term $\term$ is $(x+(y-1))$, with $var(\term) = \{x,y\}$. The type of a term is given by:
\vspace{8px}\\
$\begin{array}{lll}\sortFunction: \term \mapsto & \sort       & \mathit{if}\: \term = x \in \Variables^\sort \\ 
 & \sort_{n+1} & \mathit{if}\: \term = \function(\term_1 ... \term_n) \mathit{\:and\:} \sortFunction(\function) = \sort_1 ... \sort_{n+1}\mathit{,\:provided\:} \sortFunction(\term_i) = \sort_i
\end{array}$
\vspace{8px}\\
The set of terms with return types $\mathbb{U}^{bool}$, is denoted as $\BooleanTerms(\Variables)$. An example is $(x < y)$, where the result is $\mathit{true}$ or $\mathit{false}$.

A \textit{term-mapping} is a function $\termMapping:\Variables \rightarrow \Terms(\Variables)$\newnot{symbol:termMapping}. A \textit{valuation} $\valuation$\newnot{symbol:valuation} is a function $\valuation:\Variables \rightarrow \mathbb{U}$ that assigns constants to variables. For example, given an algebra, $\valuation:\{(x \mapsto 1), (y \mapsto 2))\}$ assigns the constants 1 and 2 to the variables $x$ and $y$ respectively.
A valuation of a term given $\Algebra$ is defined by $\valuation(\function(\term_1 ... \term_n)) \mapsto \function_\Algebra(\valuation(\term_1) ... \valuation(\term_n))$. When every variable in a term is defined by a valuation, the term can be valuated to a constant. Therefore, when every variable in a term-mapping is defined by a valuation, a new valuation can be obtained. Formally, this is defined as: $\valuation_{after}:(\Variables \rightarrow \Terms(\Variables)) \rightarrow (\Variables \rightarrow \mathbb{U})$.
