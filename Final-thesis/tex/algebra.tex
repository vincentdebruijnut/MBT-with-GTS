\section{Algebra}\label{sec:algebra}

Some basic concepts from algebra are described here. For a general introduction into logic we refer to~\cite{Huth:logic}.

A \textit{multi-sorted signature} $\langle S, F\rangle$ describes the function symbols and sorts of a formal language. $F$ is a set of function symbols, e.g. '+', '*', '=', '<', '0', '1'. Each $f\in F$ has an arity $n \in \mathbb{N}$, where a function symbol with arity $n = 0$ is called a constant symbol. The sort of a function symbol $f \in F$ is given by $\sigma(f) = S_1 ... S_n$. 

An \textit{algebra} $\mathcal{A} = \langle \mathbb{U}, \mathcal{F}\rangle$ has a non-empty set $\mathbb{U}$ of constants called a \textit{universe} and a set $\mathcal{F}$ of functions. A function $f_\mathcal{A}$ is typed $\mathbb{U}_1^{S_1} \times ... \mathbb{U}_{n-1}^{S_{n-1}} \rightarrow \mathbb{U}_n^{S_n}$, where $S$ is the sort of the function symbol given by the signature. For example, $<_\mathcal{A}: \mathbb{U}_\mathcal{A}^{int} \times \mathbb{U}_\mathcal{A}^{int} \rightarrow \mathbb{U}_\mathcal{A}^{bool}$ represents the 'less-than' comparison of two integers.
 
We define $\mathcal{V} = \mathcal{V}^{int} \sqcup \mathcal{V}^{real} \sqcup \mathcal{V}^{bool} \sqcup \mathcal{V}^{string}$ to be the set of \textit{variables}. \textit{Terms} over $V$, denoted $\mathcal{T}(V)$, are built from function symbols $F$ and variables $V \subseteq \mathcal{V}$. The definition of a term is:
\vspace{8px}\\
$\begin{array}{lrl}t & ::= & f(t_1 ... t_n) \\ & | & x\end{array}$, where $x$ is a constant.
\vspace{8px}\\
We write $var(t)$ to denote the set of variables appearing in a term $t \in \mathcal{T}(V)$. Terms $t\in \mathcal{T}(\emptyset)$ are called ground terms. An example of a term $\mathit{t}$ is $(x+y)$, with $var(t) = \{x,y\}$. The type of a term is given by:
\vspace{8px}\\
$\begin{array}{lll}\sigma: t \mapsto & s       & \mathit{if}\: t = x \in \mathcal{V}^S \\ 
                                     & s_{n+1} & \mathit{if}\: t = f(t_1 ... t_n) \mathit{\:and\:} \sigma(f) = S_1 ... S_{n+1}\mathit{,\:provided\:} \sigma(t_i) = S_i
\end{array}$
\vspace{8px}\\
A term with type $\mathbb{U}^{bool}$, is denoted as $\mathcal{R}(\mathcal{V})$. An example is $(x < y)$, where the result is $\mathit{true}$ or $\mathit{false}$.

A \textit{term-mapping} is a function $\sigma:\mathcal{V} \rightarrow \mathcal{T}(\mathcal{V})$. A \textit{valuation} $\nu$ is a function $\nu:\mathcal{V} \rightarrow \mathbb{U}$ that assigns constants to variables. For example, given an algebra, $\nu:\{(x \mapsto 1), (y \mapsto 2))\}$ assigns the constants 1 and 2 to the variables $x$ and $y$ respectively.
A valuation of a term given $\mathcal{A}$ is defined by $\nu(f(t_1 ... t_n)) \mapsto f_\mathcal{A}(\nu(t_1) ... \nu(t_n))$.
