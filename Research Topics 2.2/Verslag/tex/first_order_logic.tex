\section{First Order Logic}\label{sec:first_order_logic}

Some basic concepts from first order logic are described here, used in the definitions of section~\ref{sec:symbolic}. For a general introduction into logic we refer to~\cite{Huth:logic}.

\begin{definition}
A logical signature $\mathcal{S}$ is a tuple $\langle F, P \rangle$, where:
\begin{itemize}
 \item F is a set of \textit{function symbols}. Each $f\in F$ has a corresponding arity $n\in \mathcal{N}$. if $n = 0$ we call $f$ a constant.
 \item P is a set of \textit{predicate symbols}. Each $p\in P$ has a corresponding arity $n > 0$.
\end{itemize}
\end{definition}
 
Let $\mathcal{V}$ be a set of \textit{variables}. \textit{Terms} over $V$, denoted $\mathcal{T}(V)$, are built from function symbols $F$ and variables $V \subset \mathcal{V}$. We write $var(t)$ to denote the set of variables appearing in a term $t$. Terms $t\in \mathcal{T}(\emptyset)$ are called ground terms.

A \textit{term-mapping} is a function $\sigma:\mathcal{V} \mapsto \mathcal{T}(\mathcal{V}). For sets $V$, $W$ with $V \cup W \subset \mathcal{V}$, we write $\mathcal{T}(W )X$ for the set of term-mappings that assign to each variable $v∈V$ a term $t\in \mathcal{T}(W)$, and to each variable $v \not\in V$ the term $v$.

A \textit{valuation} $\Sigma$ is a function $\Sigma:\mathcal{V} \mapsto \mathcal{U}$, where $\mathcal{U}$ is a non-empty set called a \textit{universe}. An example of a universe is $\mathcal{N}$, the non-zero integers.

An \textit{evaluation} of $\mathcal{T}(\mathcal{V})$ is given by $\epsilon:($\mathcal{T}(\mathcal{V})$,\Sigma:\mathcal{V}) \mapsto \mathcal{U}$. An evaluation $\epsilon:\mathcal{T}(\mathcal{V}),\Sigma_before:\mathcal{V}) of the terms in a term-mapping results in a set $X \subset \mathcal{U}$. A new valuation can then be constructed by: $\Sigma_after:\mathcal{V} \mapsto X$.
